% ====================================================================
%  Main Document - Variable‑Shape Linear Algebra
% ====================================================================
\documentclass[11pt]{article}

% --------------------------------------------------------------------
%  Preamble
% --------------------------------------------------------------------
% ====================================================================
%  Preamble for Variable‑Shape Linear Algebra Paper
% ====================================================================

% --------------------------------------------------------------------
%  Packages and global layout tweaks
% --------------------------------------------------------------------
\usepackage[a4paper,margin=1in]{geometry}
\usepackage{amsmath,amssymb,mathtools}
\usepackage{amsthm}
\usepackage{titlesec}
\usepackage{needspace}
% \usepackage{microtype}  % Disabled due to font expansion issue
\usepackage{hyperref}
\usepackage{enumitem}
\usepackage{graphicx}
\usepackage{booktabs}
\usepackage{array}
\usepackage{algorithm}
\usepackage{algorithmic}
\usepackage{xcolor}
\usepackage{tcolorbox}
\usepackage{tikz}
\usetikzlibrary{positioning,arrows.meta,shapes.geometric,patterns}
\usepackage{url}

% Define colors for boxes
\definecolor{prelim}{rgb}{0.95,0.95,1.0}
\definecolor{api}{rgb}{0.95,1.0,0.95}
\definecolor{memory}{rgb}{1.0,0.95,0.95}
\raggedbottom                    % suppress large vertical glue
\allowdisplaybreaks[2]           % gentle math page‑breaks

% --------------------------------------------------------------------
%  Theorem‑like environments
% --------------------------------------------------------------------
\newtheorem{theorem}{Theorem}[section]
\newtheorem{proposition}[theorem]{Proposition}
\newtheorem{lemma}[theorem]{Lemma}
\newtheorem{definition}[theorem]{Definition}
\newtheorem{example}[theorem]{Example}

% QED symbol configuration
\renewcommand{\qedsymbol}{$\square$}
\renewcommand{\proofname}{\textit{Proof}}

% --------------------------------------------------------------------
%  Custom Commands
% --------------------------------------------------------------------

% Sets / domains
\newcommand{\D}{D}

% Shape & dimension operators
\newcommand{\shape}{\operatorname{shape}}
\newcommand{\vdim}{\operatorname{vdim}}
\newcommand{\rankop}{\operatorname{rank}}
\newcommand{\amb}{\operatorname{amb}}
\newcommand{\prom}{\operatorname{prom}}
\newcommand{\minrep}{\operatorname{minrep}}

% Structural constructors
\newcommand{\Stack}{\mathcal{S}}
\newcommand{\Wstack}{\mathcal{W}}
\newcommand{\Pyr}{\mathcal{P}}

% Transforms / reindexing
\newcommand{\perm}{\operatorname{perm}}
\newcommand{\reshape}{\operatorname{reshape}}
\newcommand{\scatter}{\operatorname{scatter}}
\newcommand{\gather}{\operatorname{gather}}
\newcommand{\unstack}{\operatorname{unstack}}
\newcommand{\unprom}{\operatorname{unprom}}

% AD helpers
\newcommand{\rev}{\operatorname{rev}}

% Sparsity / complexity
\newcommand{\nnz}{\operatorname{nnz}}
\newcommand{\cost}{\operatorname{cost}}

% Dimension law operators
\newcommand{\dimlaw}{\operatorname{dimlaw}}

% Legacy macros
\newcommand{\keywords}[1]{\textbf{Keywords:} #1}
\newcommand{\msc}[1]{\textbf{MSC:} #1}

% --------------------------------------------------------------------
%  Title information
% --------------------------------------------------------------------
\title{Variable‑Shape Linear Algebra: Mathematical Foundations and High-Performance Implementation}
\author{Royce Birnbaum}
\date{July 19, 2025}

% ====================================================================
\begin{document}
\maketitle

% ================================================================
%  Abstract
% ================================================================
\begin{abstract}
Variable‑Shape Linear Algebra (VSLA) treats dimension as intrinsic data, enabling automatic shape promotion while preserving algebraic structure. This paper formalizes VSLA through equivalence classes of finite‑dimensional vectors and develops two complete semiring instantiations: convolution and Kronecker products. We introduce the stacking operator $\Stack$ that builds higher-rank tensors from variable-shape collections, forming tensor pyramids for streaming applications. The mathematical foundation builds on classical polynomial ring theory and formal power series \cite{VonZurGathenGerhard2013}, while the dual semiring approach extends traditional matrix algebra \cite{HornJohnson2012,GondranMinoux2008}. While Cheng's Semi-Tensor Product (STP) first generalized matrix products for mismatched dimensions \cite{Cheng2001}, VSLA offers a distinct, complementary approach founded on equivalence classes and dual semiring models, leading to a sparse-by-design memory model. Complexity analysis shows FFT convolution preserves $\mathcal{O}(mn d_{\max} \log d_{\max})$ efficiency using established fast multiplication algorithms \cite{CooleyTukey1965}. Unlike existing ragged tensor frameworks \cite{TF2024,PyTorch2023,NestedTensorPyTorch2022}, VSLA provides mathematically rigorous structures with provable algebraic identities, enabling principled computation for adaptive AI architectures, sensor fusion, and scientific computing.
\end{abstract}

\vspace{0.5em}
\noindent\keywords{Variable‑shape tensors, stacking operator, tensor pyramids, semiring algebra, automatic differentiation, high‑performance computing, adaptive neural networks, sensor fusion, sparse computing, Semi-Tensor Product}

\vspace{1em}
\noindent\msc{15A69, 68W30, 65F05, 16Y60}

% ================================================================
%  Sections
% ================================================================

% ================================================================
%  Introduction
% ================================================================
\section{Context and Motivation}
\subsection{The Dimension Problem}
Traditional linear algebra fixes dimensions \(m,n\) \emph{a priori}.  Contemporary challenges—adaptive neural networks, multi‑resolution signal analysis, dynamic meshes—demand structures whose shapes evolve in real time. This fundamental mismatch between static mathematical frameworks and dynamic computational needs creates significant barriers to progress in emerging fields.

\vspace{0.5em}
\textbf{Running Example:} Consider training a convolutional neural network where filter widths adapt dynamically based on input complexity. A standard $3 \times 3$ convolution kernel $K_1 = [1, -1, 2]$ might expand to $K_2 = [1, -1, 2, 0, 1]$ for high-resolution features. Traditional frameworks require manual padding: $K_1' = [1, -1, 2, 0, 0]$ before operations, losing semantic information and incurring unnecessary computation on artificial zeros.

The core mathematical challenge lies in reconciling algebraic rigor with computational flexibility. While prior work such as the Semi-Tensor Product (STP) of matrices \cite{Cheng2001} provided a powerful mechanism for multiplying dimension-mismatched matrices via Kronecker-based lifting, VSLA arises from a different motivation: defining a rigorous algebraic system for variable-dimension objects themselves. VSLA's equivalence-class abstraction avoids materializing padded zeros and supplies dual semiring models oriented toward convolutional (polynomial) and Kronecker (tensor growth) behaviors.

\subsection{Limitations of Current Approaches}
\label{sec:limitations}

Existing frameworks fail to provide both mathematical rigor and computational flexibility:

\begin{itemize}[leftmargin=1.5em]
\item \textbf{TensorFlow Ragged Tensors:} Handle variable-length sequences but lack rigorous algebraic structure and semiring properties.
\item \textbf{PyTorch NestedTensors:} Provide dynamic shapes but without mathematical guarantees or efficient sparse representations.
\item \textbf{Manual zero-padding:} Obscures mathematical structure, wastes computation, and lacks provable algebraic identities.
\end{itemize}

\subsection{The VSLA Solution}
VSLA incorporates the shape directly into every algebraic object through mathematically rigorous equivalence classes.  Operations such as addition or convolution implicitly coerce operands to a common dimension while preserving sparsity and algebraic properties. In our example, $K_1 \oplus K_2 = [2, -2, 4, 0, 1]$ automatically, with provable semiring laws and efficient sparse computation.

\textbf{Core Innovation:} Rather than forcing all tensors into fixed dimensions, VSLA treats dimension as intrinsic data through equivalence classes $(d,v) \sim (d',v')$ when their zero-padded extensions match. This enables automatic shape promotion: $(3,[1,-1,2]) + (5,[3,0,-1,1,2]) = (5,[4,-1,1,1,2])$ without explicit padding operations.

\textbf{Mathematical Rigor:} VSLA operations form semiring structures enabling algebraic manipulation while preserving computational efficiency. Two instantiations provide different computational trade-offs: convolution semiring ($\mbox{$\mathcal{O}(mn d_{\max} \log d_{\max})$}$) and Kronecker semiring ($\mbox{$\mathcal{O}(mn d_{\max}^2)$}$).

\textbf{Practical Impact:} Variable-shape computation eliminates memory waste from artificial padding, enables adaptive neural architectures, and provides mathematical foundations for emerging applications in multi-sensor fusion, quantum simulation, and edge computing.

\subsection{Roadmap}
This paper proceeds as follows: \S\ref{sec:prelim} establishes mathematical preliminaries; \S\ref{sec:foundations}–\S\ref{sec:modelB} develop two semiring models with complete proofs; \S\ref{sec:stacking} introduces the stacking operator and tensor pyramid constructions; \S\ref{sec:vsla}–\S\ref{sec:implementation} bridge theory to implementation; \S\ref{sec:advanced_ops} extends to advanced sparse simulation operations; \S\ref{sec:evaluation}–\S\ref{sec:related} provide empirical validation and context.

% ================================================================
%  Mathematical Foundations
% ================================================================
\section{Mathematical Preliminaries}
\label{sec:prelim}

\begin{tcolorbox}[colback=prelim,colframe=blue!50!black,title=Key Definitions]
\begin{description}[leftmargin=2em]
\item[Dimension-aware vector:] An equivalence class $[(d,v)]$ where $d \in \mathbb{N}$ is the logical dimension and $v \in \mathbb{R}^d$ is the data vector.
\item[Zero-padding equivalence:] $(d_1,v) \sim (d_2,w)$ iff their extensions to $\max(d_1,d_2)$ dimensions are equal.
\item[Shape-semiring:] A semiring $S$ with dimension function $\vdim: S \to \mathbb{N}$ satisfying $\vdim(x+y) \leq \max(\vdim x, \vdim y)$ and for Model A (convolution), $\vdim(xy) = \vdim x + \vdim y - 1$, while for Model B (Kronecker), $\vdim(xy) = \vdim x \cdot \vdim y$.
\item[Variable-shape operation:] An operation that automatically promotes operands to compatible shapes before computation.
\end{description}
\end{tcolorbox}

\vspace{1em}

\begin{table}[h]
\centering
\caption{Notation Table}
\begin{tabular}{cl}
\toprule
\textbf{Symbol} & \textbf{Meaning} \\
\midrule
$D$ & Set of dimension-aware vectors \\
$[(d,v)]$ & Equivalence class of vector $v \in \mathbb{R}^d$ \\
$\vdim x$ & Logical dimension of element $x$ (also: degree) \\
$\iota_{m \to n}$ & Zero-padding map from $\mathbb{R}^m$ to $\mathbb{R}^n$ \\
$\oplus, \otimes_c$ & Addition and convolution in Model A \\
$\oplus, \otimes_K$ & Addition and Kronecker product in Model B \\
$\Stack_k$ & Stacking operator for $k$ tensors \\
$d_{\max}$ & Maximum dimension across matrix entries \\
$\mathcal{O}(\cdot)$ & Asymptotic complexity bound \\
\bottomrule
\end{tabular}
\end{table}

\begin{figure}[ht]
\centering
\begin{tikzpicture}[
  box/.style={draw, rectangle, minimum width=0.8cm, minimum height=0.6cm, font=\small},
  zero/.style={box, pattern=north east lines, pattern color=gray!80},
  value/.style={box, fill=white},
  eq/.style={font=\Large}
]

% Original vectors
\node[value] (v1) at (0,2) {1};
\node[value] (v2) at (0.8,2) {-1};
\node[value] (v3) at (1.6,2) {2};
\node[above=0.2cm of v2] {$v$};

\node[value] (w1) at (0,0.5) {3};
\node[value] (w2) at (0.8,0.5) {0};
\node[value] (w3) at (1.6,0.5) {-1};
\node[value] (w4) at (2.4,0.5) {1};
\node[value] (w5) at (3.2,0.5) {2};
\node[above=0.2cm of w3] {$w$};

% Arrow and equivalence
\node[eq] at (4.5,1.25) {$\sim$};

% Padded vectors
\node[value] (pv1) at (6,2) {1};
\node[value] (pv2) at (6.8,2) {-1};
\node[value] (pv3) at (7.6,2) {2};
\node[zero] (pv4) at (8.4,2) {0};
\node[zero] (pv5) at (9.2,2) {0};
\node[above=0.2cm of pv3] {$\iota_{3 \to 5}(v)$};

\node[value] (pw1) at (6,0.5) {3};
\node[value] (pw2) at (6.8,0.5) {0};
\node[value] (pw3) at (7.6,0.5) {-1};
\node[value] (pw4) at (8.4,0.5) {1};
\node[value] (pw5) at (9.2,0.5) {2};
\node[above=0.2cm of pw3] {$\iota_{5 \to 5}(w)$};

% Addition result
\node[eq] at (10.5,1.25) {$+$};

\node[value] (r1) at (12,1.25) {4};
\node[value] (r2) at (12.8,1.25) {-1};
\node[value] (r3) at (13.6,1.25) {1};
\node[value] (r4) at (14.4,1.25) {1};
\node[value] (r5) at (15.2,1.25) {2};
\node[above=0.2cm of r3] {$v + w$};

% Labels
\node[below=0.5cm of v2] {\footnotesize Dimension 3};
\node[below=0.5cm of w3] {\footnotesize Dimension 5};
\node[below=0.5cm of pv3] {\footnotesize Padded to 5};
\node[below=0.5cm of pw3] {\footnotesize Already 5};
\node[below=0.5cm of r3] {\footnotesize Result dimension 5};

\end{tikzpicture}
\caption{Zero-padding equivalence in VSLA. Two vectors of different dimensions become equivalent after padding to a common dimension, enabling automatic variable-shape operations. The values of the vector components are shown inside each cell. White cells contain actual values, and hatched cells represent trailing zeros.}
\label{fig:zeropadding}
\end{figure}

% ================================================================
\section{Mathematical Foundations}
\label{sec:foundations}
\subsection{The Dimension‑Aware Space}
\begin{definition}[Dimension‑Aware Vectors]\label{def:DAspace}
Define the graded set
\[
  D_e\;:=\;\bigsqcup_{d\ge0}\,\{d\}\times\mathbb R^{d},
\]
where \(\mathbb R^{0}:=\{\,[]\}\) denotes the empty vector. Throughout this paper, we use $\vdim(\cdot)$ and "degree" interchangeably to refer to the dimensional extent of elements.
\end{definition}

\begin{definition}[Zero‑Padding Equivalence]\label{def:padding}
For \(m\le n\) let \(\iota_{m\rightarrow n}\colon\mathbb R^{m}\to\mathbb R^{n}\) append \(n-m\) trailing zeros.  Put
\[
  (d_1,v)\sim(d_2,w)
  \iff \iota_{d_1\rightarrow n}(v)=\iota_{d_2\rightarrow n}(w),\quad n:=\max(d_1,d_2).
\]
\end{definition}

\begin{proposition}\label{prop:equiv}
The relation \(\sim\) is an equivalence relation, yielding the set \(D:=D_e/\!\sim\) of \emph{dimension‑aware vectors}. This construction relates to classical quotient structures in commutative algebra \cite{AtiyahMacdonald1969,Eisenbud1995}, though adapted for computational rather than geometric applications.
\end{proposition}
\begin{proof}
Reflexivity and symmetry are immediate from Definition\;\ref{def:padding}.  For transitivity pad to \(n:=\max(d_1,d_2,d_3)\).\end{proof}

\subsection{Additive Structure}
\begin{theorem}\label{thm:add}
\(\bigl(D,+,0\bigr)\) is a commutative monoid where
\[
  \bigl[(d_1,v)\bigr]+\bigl[(d_2,w)\bigr]
  :=\bigl[(n,\,\iota_{d_1\rightarrow n}(v)+\iota_{d_2\rightarrow n}(w))\bigr],
  \quad n:=\max(d_1,d_2),\qquad
  0:=\bigl[(0,[])\bigr].
\]
\end{theorem}
\begin{proof}
Well‑definedness follows from Proposition\;\ref{prop:equiv}. Associativity and commutativity inherit from \(\mathbb R^n\).\end{proof}

\subsection{Shape and Promotion Operators}

\begin{definition}[Shape Operator]
For a VSLA tensor $x \in \mathbb{T}_r$, define the \emph{shape operator}
\[
\shape(x) = (n_1, \ldots, n_r) \in \mathbb{N}^r
\]
where $(n_1, \ldots, n_r)$ are the dimensions of any representative $(d, v) \in x$ with $d = (n_1, \ldots, n_r)$.
\end{definition}

\begin{definition}[Ambient Shape Operator]
For a collection of tensors $A^{(1)}, \ldots, A^{(k)} \in \mathbb{T}_r$, define
\[
\amb(A^{(1)}, \ldots, A^{(k)}) := \left(\max_i \shape(A^{(i)})_1, \ldots, \max_i \shape(A^{(i)})_r\right).
\]
\end{definition}

\begin{definition}[Promotion Operator]
For a tensor $x \in \mathbb{T}_r$ and target shape $\mathbf{n} = (n_1, \ldots, n_r)$, define
\[
\prom_{\mathbf{n}}(x) := [(n_1 \times \cdots \times n_r, \iota_{\shape(x) \rightarrow \mathbf{n}}(v))]
\]
where $v$ is any representative vector of $x$ and $\iota$ is the canonical zero-padding embedding.
\end{definition}

\begin{definition}[Minimal Representative Operator]
For a tensor $x \in \mathbb{T}_r$, define
\[
\minrep(x) := \text{unique representative } (d,v) \in x \text{ with no trailing zero slices}.
\]
\end{definition}

% ================================================================
%  Model A: The Convolution Semiring
% ================================================================
\section{Model A: The Convolution Semiring}
\label{sec:modelA}

\noindent \textbf{Notation:} Throughout this section, we use $\vdim(x)$ and ``degree'' interchangeably when referring to the dimensional extent of VSLA elements, particularly in the context of the polynomial isomorphism (Theorem~\ref{thm:polyIso}) and graded ring discussion that follows.

\subsection{Convolution Product}
\begin{definition}
For \(v\in\mathbb R^{d_1}\) and \(w\in\mathbb R^{d_2}\) define the discrete convolution
\[
  (v\ast w)_k \;:=\;\sum_{i+j=k+1} v_i\,w_j,\qquad k=0,\dots,d_1+d_2-2.
\]
Put
\[
  \bigl[(d_1,v)\bigr]\otimes_c\bigl[(d_2,w)\bigr]
  :=\begin{cases}
       0, & d_1d_2=0, \\
       \bigl[(d_1+d_2-1,\,v\ast w)\bigr], & \text{otherwise.}
     \end{cases}
\]
\end{definition}


\begin{theorem}\label{thm:convSemiring}
\(\bigl(D,+,\otimes_c,0,1\bigr)\) is a commutative semiring with \(1:=\bigl[(1,[1])\bigr]\).
\end{theorem}
\begin{proof}
We verify the semiring axioms:

\textit{Associativity of $\otimes_c$:}
For $a, b, c \in D$ with representatives $[(d_1,u)], [(d_2,v)], [(d_3,w)]$, we need $(a \otimes_c b) \otimes_c c = a \otimes_c (b \otimes_c c)$. By definition, $(u \ast v) \ast w$ and $u \ast (v \ast w)$ both equal
\[
\sum_{i+j+k=n+2} u_i v_j w_k
\]
when expanding the convolution index arithmetic. Thus both products have degree $d_1 + d_2 + d_3 - 2$ and identical coefficients.

\textit{Commutativity of $\otimes_c$:}
The convolution $(u \ast v)_k = \sum_{i+j=k+1} u_i v_j = \sum_{i+j=k+1} v_j u_i = (v \ast u)_k$ by symmetry of the index condition.

\textit{Distributivity:}
For $a, b, c \in D$, we have $a \otimes_c (b + c) = a \otimes_c b + a \otimes_c c$ since convolution distributes over pointwise addition: $u \ast (v + w) = u \ast v + u \ast w$ coefficientwise.

\textit{Identity elements:}
The zero element $0 = [(0,[])]$ satisfies $0 \otimes_c a = 0$ by the first case in the definition. The one element $1 = [(1,[1])]$ satisfies $(1 \ast v)_k = v_k$ for all $k$, making it the multiplicative identity.
\end{proof}

\begin{theorem}[Polynomial Isomorphism]\label{thm:polyIso}
The map \(\Phi\bigl([(d,v)]\bigr):=\sum_{i=0}^{d-1} v_{i+1}\,x^{i}\) is a semiring isomorphism \(D\cong\mathbb R[x]\).
\end{theorem}
\begin{proof}
We verify that $\Phi$ is a well-defined semiring homomorphism, then show bijectivity.

\textit{Well-definedness:}
If $[(d_1,v)] = [(d_2,w)]$, then after padding to $n = \max(d_1,d_2)$, we have $\iota_{d_1 \to n}(v) = \iota_{d_2 \to n}(w)$. This means $v_i = w_i$ for $i = 1,\ldots,\min(d_1,d_2)$ and the remaining components are zero. 

\textit{Indexing Note:} We use 1-based indexing for vector components ($v_i$ with $i \geq 1$) and 0-based indexing for polynomial powers ($x^j$ with $j \geq 0$). The mapping $\Phi([(d,v)]) = \sum_{i=0}^{d-1} v_{i+1} x^i$ bridges these conventions: the $i$-th polynomial coefficient comes from the $(i+1)$-th vector component.

Thus $\Phi([(d_1,v)]) = \sum_{i=0}^{d_1-1} v_{i+1} x^i = \sum_{i=0}^{d_2-1} w_{i+1} x^i = \Phi([(d_2,w)])$.

\textit{Additive homomorphism:}
For $a = [(d_1,v)], b = [(d_2,w)]$ with $n = \max(d_1,d_2)$:
\begin{align}
\Phi(a + b) &= \Phi([(n, \iota_{d_1 \to n}(v) + \iota_{d_2 \to n}(w))]) \\
&= \sum_{i=0}^{n-1} (\iota_{d_1 \to n}(v)_{i+1} + \iota_{d_2 \to n}(w)_{i+1}) x^i \\
&= \sum_{i=0}^{n-1} \iota_{d_1 \to n}(v)_{i+1} x^i + \sum_{i=0}^{n-1} \iota_{d_2 \to n}(w)_{i+1} x^i \\
&= \Phi(a) + \Phi(b)
\end{align}

\textit{Multiplicative homomorphism:}
For convolution $a \otimes_c b = [(d_1+d_2-1, v \ast w)] $:
\begin{align}
\Phi(a \otimes_c b) &= \sum_{k=0}^{d_1+d_2-2} (v \ast w)_{k+1} x^k \\
&= \sum_{k=0}^{d_1+d_2-2} \left(\sum_{i+j=k+1} v_i w_j\right) x^k \\
&= \sum_{i=1}^{d_1} \sum_{j=1}^{d_2} v_i w_j x^{i+j-2} \\
&= \left(\sum_{i=0}^{d_1-1} v_{i+1} x^i\right)\left(\sum_{j=0}^{d_2-1} w_{j+1} x^j\right) \\
&= \Phi(a) \cdot \Phi(b)
\end{align}

\textit{Surjectivity:}
Every polynomial $p(x) = \sum_{i=0}^{d-1} a_i x^i \in \mathbb{R}[x]$ equals $\Phi([(d, (a_0, a_1, \ldots, a_{d-1}))])$.

\textit{Injectivity:}
If $\Phi([(d_1,v)]) = \Phi([(d_2,w)])$, then the polynomials have identical coefficients, so after padding both vectors have the same components, hence $[(d_1,v)] = [(d_2,w)]$.
\end{proof}

\subsection{Relation to Graded Rings and Rees Algebras}
The convolution semiring \((D, +, \otimes_c)\) has deep connections to established algebraic structures, particularly graded rings and the Rees algebra.

\textbf{Graded Ring Structure:}
The polynomial isomorphism \(\Phi: D \cong \mathbb{R}[x]\) (Theorem~\ref{thm:polyIso}) reveals that the convolution semiring is isomorphic to the polynomial ring \(\mathbb{R}[x]\), which is a classic example of a graded ring. A ring \(R\) is graded if it can be written as a direct sum \(R = \bigoplus_{n \ge 0} R_n\) such that \(R_m R_n \subseteq R_{m+n}\). For \(\mathbb{R}[x]\), the graded components are the subspaces of homogeneous polynomials of degree \(n\), i.e., \(R_n = \text{span}(x^n)\). The VSLA "degree" corresponds directly to the polynomial degree, and the convolution operation corresponds to polynomial multiplication, which respects the grading.

\textbf{Rees Algebra:}
The Rees algebra (or blow-up algebra) associated with an ideal \(I\) of a ring \(R\) is defined as \(R(I) = \bigoplus_{n \ge 0} I^n t^n \subseteq R[t]\). While not a direct equivalent, the VSLA framework shares the spirit of the Rees algebra by tracking "powers" of some fundamental structure. In VSLA, the "ideal" can be thought of as the space of all possible vector extensions, and the "degree" of a VSLA element tracks how "large" an extension is needed. The key difference is that VSLA is built on a semiring and focuses on computational aspects of variable-length data, whereas the Rees algebra is a tool in commutative algebra for studying the structure of ideals.

By framing VSLA in this context, we see that it is not an ad-hoc construction but rather a computationally-oriented realization of well-understood algebraic objects, tailored to the specific challenges of variable-shape data in high-performance computing.

\begin{theorem}[Completion]\label{thm:completion}
Equip \(D\) with the norm \(\lVert[(d,v)]\rVert_1:=\sum_{i=1}^{d}|v_i|\). The Cauchy completion of \((D, \lVert \cdot \rVert_1)\) is isometrically isomorphic to the Banach space \(\ell^1(\mathbb{N}_0)\) of absolutely summable sequences, which is a subspace of the formal power series ring \(\mathbb R[[x]]\). This extends classical results on polynomial ring completions \cite{Rudin1991,VonZurGathenGerhard2013} to the variable-dimension setting.
\end{theorem}
\begin{proof}
We provide a complete proof with topological clarification.

\textit{1. The Normed Space:}
The map \(\lVert \cdot \rVert_1: D \to \mathbb{R}_{\ge 0}\) is a well-defined norm. The isomorphism \(\Phi: D \to \mathbb{R}[x]\) from Theorem~\ref{thm:polyIso} allows us to view \(D\) as the space of polynomials. For a polynomial \(p(x) = \sum_{i=0}^{d-1} a_i x^i\), the norm is \(\lVert p \rVert_1 = \sum_{i=0}^{d-1} |a_i|\). This is the standard \(\ell^1\)-norm on the coefficients.

\textit{2. Cauchy Sequences:}
Let \((f_n)_{n \in \mathbb{N}}\) be a Cauchy sequence in \(D\). Via \(\Phi\), this corresponds to a Cauchy sequence of polynomials \((p_n)_{n \in \mathbb{N}}\) in \((\mathbb{R}[x], \lVert \cdot \rVert_1)\). For any \(\epsilon > 0\), there exists \(N\) such that for \(n,m > N\), \(\lVert p_n - p_m \rVert_1 < \epsilon\).
Let \(p_n(x) = \sum_{i=0}^{\vdim(p_n)} a_{n,i} x^i\). The norm condition implies that for each fixed coefficient index \(i\), the sequence of real numbers \((a_{n,i})_{n \in \mathbb{N}}\) is a Cauchy sequence in \(\mathbb{R}\). This is because \(|a_{n,i} - a_{m,i}| \le \sum_j |a_{n,j} - a_{m,j}| = \lVert p_n - p_m \rVert_1 < \epsilon\).

\textit{3. The Limit Object:}
Since \(\mathbb{R}\) is complete, each sequence \((a_{n,i})_n\) converges to a limit \(a_i \in \mathbb{R}\). We define the limit object in the completion as the formal power series \(p(x) = \sum_{i=0}^{\infty} a_i x^i\).

\textit{4. The Completion Space:}
We must show that this limit object \(p(x)\) is in the space \(\ell^1(\mathbb{N}_0)\), i.e., \(\sum_{i=0}^{\infty} |a_i| < \infty\).
Since \((p_n)\_n\) is a Cauchy sequence, it is bounded: there exists \(M > 0\) such that \(\lVert p_n \rVert_1 \le M\) for all \(n\). For any finite \(K\), \(\sum_{i=0}^K |a_{n,i}| \le M\). Taking the limit as \(n \to \infty\), we get \(\sum_{i=0}^K |a_i| \le M\). Since this holds for all \(K\), the series \(\sum_{i=0}^{\infty} |a_i|\) converges absolutely with \(\sum_{i=0}^{\infty} |a_i| \le M < \infty\), confirming that \(p(x)\) is in \(\ell^1(\mathbb{N}_0)\).

\textit{5. Isomorphism:}
The completion of \((\mathbb{R}[x], \lVert \cdot \rVert_1)\) is the Banach space \(\ell^1(\mathbb{N}_0)\). The map \(\Phi\) extends to an isometric isomorphism from the completion of \(D\) to \(\ell^1(\mathbb{N}_0)\). This space is a well-known Banach algebra under convolution, and it forms a proper subalgebra of \((\mathbb R[[x]], +, \cdot)\) under the coefficientwise convergence topology (whereas \(\mathbb R[[x]]\) with its natural topology is complete under a different, non-normable topology).
\end{proof}

\begin{lemma}[Commutativity Comparison]
Model A is a commutative semiring; Model B is non-commutative except when one operand is scalar.
\end{lemma}

% ================================================================
%  Model B: The Kronecker Semiring
% ================================================================
\section{Model
B: The Kronecker Semiring}
\label{sec:modelB}
\subsection{Kronecker Product}
\begin{definition}
For \(v\in\mathbb R^{d_1}\), \(w\in\mathbb R^{d_2}\), let
\[v\otimes_K w := (v_1w_1,\dots,v_1w_{d_2},\,v_2w_1,\dots,v_{d_1}w_{d_2}).\]
Define
\[\bigl[(d_1,v)\bigr]\otimes_K \bigl[(d_2,w)\bigr] := \bigl[(d_1d_2,\,v\otimes_K w)\bigr].\]
\end{definition}

\begin{theorem}
\(\bigl(D,+,\otimes_K,0,1\bigr)\) is a non\-commutative semiring.
\end{theorem}
\begin{proof}
We verify the semiring axioms systematically.

\textit{Additive structure:} $(D,+,0)$ is already a commutative monoid by Theorem~\ref{thm:add}.

\textit{Associativity of $\otimes_K$:} For $a = [(d_1,u)]$, $b = [(d_2,v)]$, $c = [(d_3,w)] $:
\begin{align}
(a \otimes_K b) \otimes_K c &= [(d_1 d_2, u \otimes_K v)] \otimes_K [(d_3,w)] \\
&= [(d_1 d_2 d_3, (u \otimes_K v) \otimes_K w)]
\end{align}
and
\begin{align}
a \otimes_K (b \otimes_K c) &= [(d_1,u)] \otimes_K [(d_2 d_3, v \otimes_K w)] \\
&= [(d_1 d_2 d_3, u \otimes_K (v \otimes_K w)]
\end{align}
Both expressions yield vectors in $\mathbb{R}^{d_1 d_2 d_3}$ with components $(u \otimes_K v \otimes_K w)_{i,j,k} = u_i v_j w_k$ in the lexicographic order, so they are equal.

\textit{Multiplicative identity:} For $1 = [(1,[1])]$ and any $a = [(d,v)] $:
\[1 \otimes_K a = [(1 \cdot d, [1] \otimes_K v)] = [(d, (1 \cdot v_1, 1 \cdot v_2, \ldots, 1 \cdot v_d))] = [(d,v)] = a\]
Similarly, $a \otimes_K 1 = a$.

\textit{Distributivity:} For $a = [(d_1,u)]$, $b = [(d_2,v)]$, $c = [(d_2,w)] $:
\begin{align}
a \otimes_K (b + c) &= [(d_1,u)] \otimes_K [(d_2, v + w)] \\
&= [(d_1 d_2, u \otimes_K (v + w))] \\
&= [(d_1 d_2, (u_1(v_1 + w_1), \ldots, u_1(v_{d_2} + w_{d_2}), \\
&\qquad\qquad u_2(v_1 + w_1), \ldots, u_{d_1}(v_{d_2} + w_{d_2})))] \\
&= [(d_1 d_2, (u \otimes_K v) + (u \otimes_K w))] \\
&= a \otimes_K b + a \otimes_K c
\end{align}
Right distributivity follows similarly.

\textit{Absorption by zero:} $0 \otimes_K a = [(0 \cdot d, \emptyset)] = 0$ and $a \otimes_K 0 = 0$ by the definition of Kronecker product with the empty vector.

\textit{Non-commutativity:} Consider $a = [(2, (1,0))]$ and $b = [(2, (0,1))]$. Then:
\[a \otimes_K b = [(4, (1 \cdot 0, 1 \cdot 1, 0 \cdot 0, 0 \cdot 1))] = [(4, (0,1,0,0))]\]
\[b \otimes_K a = [(4, (0 \cdot 1, 0 \cdot 0, 1 \cdot 1, 1 \cdot 0))] = [(4, (0,0,1,0))]\]
Since $(0,1,0,0) \neq (0,0,1,0)$, we have $a \otimes_K b \neq b \otimes_K a$.
\end{proof}

\paragraph{Practical Implications of Non-Commutativity}
The non-commutativity of the Kronecker product is not merely a theoretical curiosity; it is a crucial feature for many applications. In quantum computing, the state of a multi-qubit system is described by the Kronecker product of single-qubit states, and the order of the product is significant. Similarly, in tensor network methods for simulating many-body quantum systems, such as Matrix Product States (MPS) or Projected Entangled Pair States (PEPS), the non-commutative nature of the Kronecker product is fundamental to capturing the entanglement structure of the system. In these domains, VSLA's Model B provides a natural and efficient framework for representing and manipulating these non-commutative structures, especially when the constituent systems have different dimensions.

\begin{proposition}\label{prop:commCase}
$x\otimes_K y = y\otimes_K x$ iff $\vdim x =1$ or $\vdim y =1$ (i.e. one operand is scalar).
\end{proposition}

\begin{lemma}[Scalar\-Commutation]\label{lem:scalarComm}
If $x=\alpha\,1$ with $\alpha\in\mathbb R$ then $x\otimes_K y = y\otimes_K x$ for all $y\in D$.
\end{lemma}
\begin{proof}
Both products equal $\alpha\,y$ by definition.\end{proof}

% ================================================================
%  Stacking Operator and Tensor Pyramids
% ================================================================
\section{Stacking Operator and Tensor Pyramids}
\label{sec:stacking}

This section introduces a major theoretical extension to VSLA: the \emph{stacking operator} $\Stack$ that builds higher-rank tensors from collections of lower-rank tensors, and the \emph{window-stacking operator} $\Wstack$ for constructing tensor pyramids from streaming data.

\subsection{The Stacking Operator $\Stack$}

\begin{definition}[Stacking Operator]
\label{def:stacking}
For $k \geq 1$, define the stacking operator
\[
\Stack_k : (\mathbb{T}_r)^k \longrightarrow \mathbb{T}_{r+1}
\]
that maps $k$ rank-$r$ tensors to a single rank-$(r+1)$ tensor by concatenation along a fresh leading axis.
\end{definition}

\begin{figure}[h]
\centering
\begin{tikzpicture}[
  box/.style={draw, rectangle, minimum width=1.5cm, minimum height=0.8cm, font=\small},
  zero/.style={box, fill=gray!20},
  value/.style={box, fill=gray!20},
  eq/.style={font=\Large}
]

% Input tensors with different shapes
\node[draw, thick, gray!90, fill=gray!70, text=white, minimum width=1.5cm, minimum height=0.8cm] (t1) at (0,2) {$A^{(1)}$};
\node[below=0.1cm of t1, gray!90] {$2 \times 3$};

\node[draw, thick, gray!90, fill=gray!70, text=white, minimum width=2cm, minimum height=0.8cm] (t2) at (0,0.5) {$A^{(2)}$};
\node[below=0.1cm of t2, gray!90] {$2 \times 4$};

\node[draw, thick, gray!90, fill=gray!70, text=white, minimum width=1cm, minimum height=0.8cm] (t3) at (0,-1) {$A^{(3)}$};
\node[below=0.1cm of t3, gray!90] {$2 \times 2$};

% Arrow
\draw[->, thick] (2.5,0.5) -- (4.5,0.5);
\node[above] at (3.5,0.5) {$\Stack_3$};

% Ambient shape calculation
\node[align=center] at (6.5,3) {\textbf{1. Compute ambient shape:} $\mathbf{n} = (\max\{2,2,2\}, \max\{3,4,2\}) = (2,4)$};

% Padded tensors
\node[draw, thick, orange!90, fill=orange!70, text=white, minimum width=2cm, minimum height=0.8cm] (p1) at (6.5,1.5) {$A^{(1)}$ padded};
\node[below=0.1cm of p1, orange!80] {$2 \times 4$};

\node[draw, thick, orange!90, fill=orange!70, text=white, minimum width=2cm, minimum height=0.8cm] (p2) at (6.5,0.5) {$A^{(2)}$ (no pad)};
\node[below=0.1cm of p2, orange!80] {$2 \times 4$};

\node[draw, thick, orange!90, fill=orange!70, text=white, minimum width=2cm, minimum height=0.8cm] (p3) at (6.5,-0.5) {$A^{(3)}$ padded};
\node[below=0.1cm of p3, orange!80] {$2 \times 4$};

% Final stacked tensor
\draw[->, thick] (8.5,0.5) -- (10.5,0.5);
\node[above] at (9.5,0.5) {\textbf{2. Stack}};

\node[draw, thick, red!90, fill=red!70, text=white, minimum width=2cm, minimum height=2.4cm] (result) at (12,0.5) {
\begin{tabular}{c}
$A^{(1)}$ \\
\hline
$A^{(2)}$ \\
\hline
$A^{(3)}$
\end{tabular}
};
\node[below=0.1cm of result, red!90] {$3 \times 2 \times 4$};
\node[above=0.1cm of result, red!90] {\textbf{Result}};

% Add labels for zero-padding
\node[align=center, gray!70] at (4,-2.5) {\footnotesize Zero-padding applied\\
to reach ambient shape};
\draw[gray!70, dashed, thick] (t1) -- (4,-1.8);
\draw[gray!70, dashed, thick] (t3) -- (4,-1.8);

\end{tikzpicture}
\caption{Stacking operator $\Stack_3$ applied to three variable-shape matrices. The operator computes the ambient shape (element-wise maximum dimensions), applies zero-padding equivalence to achieve uniform shapes, then concatenates along a new leading axis to form a higher-rank tensor.}
\label{fig:stacking}
\end{figure}

The construction proceeds as follows. Given representatives
\[
A^{(1)}, \ldots, A^{(k)} \in \mathbb{K}^{n_1^{(i)} \times \cdots \times n_r^{(i)}},
\]
compute the ambient shape and promote each tensor:
\[
\amb(A^{(1)}, \ldots, A^{(k)}) := \left(\max_i n_1^{(i)}, \ldots, \max_i n_r^{(i)}\right),
\]
\[
\hat{A}^{(i)} := \prom_{\amb}(A^{(i)}).
\]
Then the stacking operator is defined as:
\[
\Stack_k(A^{(1)}, \ldots, A^{(k)})_{i,\mathbf{j}} = \hat{A}^{(i)}_{\mathbf{j}}.
\]

For $k = 0$, define $\Stack_0 \coloneqq 0_{\mathbb{T}_{r+1}}$ (the neutral element).

\begin{example}[2D Matrix Stacking]
Consider stacking two matrices: $A^{(1)} = \begin{bmatrix} 1 & 2 \\ 3 & 4 \end{bmatrix}$ and $A^{(2)} = \begin{bmatrix} 5 & 6 & 7 \\ 8 & 9 & 10 \end{bmatrix}$.

The ambient shape is $(2, 3)$. After padding $A^{(1)}$ to $(2, 3)$:
\[
\Stack_2(A^{(1)}, A^{(2)}) = \begin{bmatrix}
\begin{bmatrix} 1 & 2 & 0 \\ 3 & 4 & 0 \end{bmatrix} \[0.5em]
\begin{bmatrix} 5 & 6 & 7 \\ 8 & 9 & 10 \end{bmatrix}
\end{bmatrix}
\]
yielding a $2 \times 2 \times 3$ tensor.
\end{example}

\begin{theorem}[Algebraic Properties of Stacking]
\label{thm:stacking-properties}
The stacking operator satisfies:
\begin{enumerate}
\item \textbf{Associativity (nested levels):} $\Stack_m(\Stack_{k_1}(\mathbf{A}_1), \dots, \Stack_{k_m}(\mathbf{A}_m))$ is equivalent to $\Stack_{k_1+\dots+k_m}(\mathbf{A}_1, \dots, \mathbf{A}_m)$ after a canonical reshape.
\item \textbf{Neutral-zero absorption:} Injecting VSLA zeros anywhere in the argument list leaves the equivalence class unchanged.
\item \textbf{Distributivity over $+$ and $\odot$}: For semiring operations, stacking distributes over element-wise operations after promotion to common ambient shape.
\end{enumerate}
\end{theorem}

\begin{proof}
(1) \emph{Associativity:} We provide a more rigorous proof. Let $\mathbf{A}_i = (A^{(i)}_1, \dots, A^{(i)}_{k_i})$ be $m$ lists of rank-$r$ tensors. Let $K = \sum_{i=1}^m k_i$. 

The left-hand side is $L = \Stack_m(\Stack_{k_1}(\mathbf{A}_1), \dots, \Stack_{k_m}(\mathbf{A}_m))$. Let $\mathbf{n}_i$ be the ambient shape for each inner stack $\Stack_{k_i}(\mathbf{A}_i)$, and let $\mathbf{N}$ be the ambient shape for the outer stack. Then $\mathbf{N}_j = \max_{i=1\dots m} (\mathbf{n}_i)_j$. The resulting tensor $L$ has rank $r+2$ and shape $(m, \mathbf{N})$.

The right-hand side is $R = \Stack_K(\mathbf{A}_1, \dots, \mathbf{A}_m)$. Let the ambient shape for this combined stack be $\mathbf{N}'$. By definition, $\mathbf{N}'_j = \max_{i,j} \vdim(A^{(i)}_j) = \max_i (\max_j \vdim(A^{(i)}_j)) = \max_i (\mathbf{n}_i)_j = \mathbf{N}_j$. So the ambient shapes are identical.

The elements of $L$ are given by $(L)_{i,j,...} = (\Stack_{k_i}(\mathbf{A}_i))_{j,...}$ padded to shape $\mathbf{N}$. The elements of $R$ are given by $(R)_{l,...}$ where $l$ indexes the concatenated list of all $A$ tensors. There is a canonical mapping from the double index $(i,j)$ to the single index $l$ that preserves the order of the tensors. Since the padding is to the same ambient shape $\mathbf{N}$, the resulting tensors are identical up to a reshape of the leading axes from $(m, k_1, \dots, k_m)$ to a single flattened axis of size $K$.

(2) \emph{Neutral-zero absorption:} Zero representatives in VSLA have the form $[(d, \mathbf{0})]$ where $\mathbf{0}$ is the zero vector. In the stacked output, these contribute blocks of zeros which preserve the zero-padding equivalence relation by definition.

(3) \emph{Distributivity:} Given tensors $A^{(i)}$ and $B^{(i)}$ and operation $\circ \in \{+, \odot\}$, we have $\Stack_k(A^{(1)} \circ B^{(1)}, \ldots) = \Stack_k(A^{(1)}, \ldots) \circ \Stack_k(B^{(1)}, \ldots)$ after promoting all operands to the common ambient shape, since $\circ$ operates element-wise within blocks.
\end{proof}

\begin{proposition}[Monoidal Category Structure]
The triple $(\mathbb{T}_r, +, \Stack)$ forms a \emph{strict monoidal category} where $\Stack$ is the tensor product on objects of type ``list of rank-$r$ tensors''.
\end{proposition}

\begin{proof}
A strict monoidal category consists of:
\begin{enumerate}
    \item A category $\mathcal{C}$.
    \item A bifunctor $\otimes: \mathcal{C} \times \mathcal{C} \to \mathcal{C}$, called the tensor product.
    \item An identity object $I \in \mathcal{C}$.
    \item The tensor product is strictly associative: $(A \otimes B) \otimes C = A \otimes (B \otimes C)$.
    \item The identity object is a strict identity for the tensor product: $I \otimes A = A$ and $A \otimes I = A$.
\end{enumerate}

We define a category $\mathcal{C}_{VSLA}$ where:
\begin{itemize}
    \item \textbf{Objects:} Finite lists of rank-$r$ VSLA tensors, i.e., elements of $(\mathbb{T}_r)^k$ for any $k \ge 0$. An object is denoted $\mathbf{A} = (A_1, \dots, A_k)$.
    \item \textbf{Morphisms:} Identity maps (we focus on the categorical structure of objects and tensor product).
\end{itemize}

The monoidal product $\otimes$ is list concatenation:
\[
    \mathbf{A} \otimes \mathbf{B} = (A_1, \dots, A_k) \otimes (B_1, \dots, B_m) := (A_1, \dots, A_k, B_1, \dots, B_m).
\]
The identity object $I$ is the empty list $() \in (\mathbb{T}_r)^0$.

Verification of strict monoidal axioms:
\begin{itemize}
    \item \textbf{Associativity:} List concatenation is strictly associative. For lists $\mathbf{A}$, $\mathbf{B}$, and $\mathbf{C}$, both $(\mathbf{A} \otimes \mathbf{B}) \otimes \mathbf{C}$ and $\mathbf{A} \otimes (\mathbf{B} \otimes \mathbf{C})$ yield the same concatenated list.
    \item \textbf{Identity Laws:} The empty list acts as strict identity: $I \otimes \mathbf{A} = () \otimes \mathbf{A} = \mathbf{A}$ and $\mathbf{A} \otimes I = \mathbf{A} \otimes () = \mathbf{A}$.
\end{itemize}

Thus $(\mathcal{C}_{VSLA}, \otimes, I)$ is a strict monoidal category. The stacking operator $\Stack$ acts as a functor from this category to single VSLA tensors, with the property $\Stack_m(\Stack_k(\mathbf{A}), \Stack_k(\mathbf{B})) \cong \Stack_{mk}(\mathbf{A}, \mathbf{B})$ demonstrating compositional structure.
\end{proof}

\textbf{Practical Interpretation:} The strict monoidal category structure guarantees that stacking operations compose predictably and associatively. This means that $\Stack_2(\Stack_2(A,B),C) = \Stack_3(A,B,C)$ up to canonical isomorphism, enabling reliable nested tensor constructions in streaming applications and recursive data structures.

\subsection{Window-Stacking and Tensor Pyramids}

\begin{definition}[Window-Stacking Operator]
Let $w \in \mathbb{N}^+$ be a fixed \emph{window length}. For a stream $(X^{(0)}, X^{(1)}, \ldots) \subset \mathbb{T}_r$, define
\[
\Wstack_w\bigl(X^{(t)}\bigr)_s = \Stack_w\bigl(X^{(sw)}, \ldots, X^{(sw+w-1)}\bigr) \in \mathbb{T}_{r+1}, \quad s = 0, 1, \ldots
\]
This slides a window of length $w$ with step $w$ (non-overlapping) and stacks the contents.
\end{definition}

\begin{definition}[Tensor Pyramids]
Compose $\Wstack$ repeatedly with window sizes $w_1, w_2, \ldots, w_d$:
\[
X^{(0)} \xrightarrow{\Wstack_{w_1}} \mathbb{T}_{r+1} \xrightarrow{\Wstack_{w_2}} \mathbb{T}_{r+2} \cdots \xrightarrow{\Wstack_{w_d}} \mathbb{T}_{r+d}
\]
Each level aggregates lower-level tensors into the next rank, giving a \emph{$d$-level tensor pyramid}.
\end{definition}

\textbf{Connection to Classical Pyramids:} VSLA tensor pyramids generalize classical pyramid structures from signal processing and computer vision. Like Gaussian pyramids that progressively blur and downsample images, or Laplacian pyramids that capture multi-scale edge information, tensor pyramids create hierarchical representations. However, unlike these fixed-resolution approaches, VSLA tensor pyramids handle variable-shape data at each level through the zero-padding equivalence relation, enabling adaptive multi-resolution processing without predetermined scale factors or uniform downsampling ratios.

\begin{example}[Signal Processing Pyramid]
Consider a 1D signal stream with window sizes $w_1 = 4, w_2 = 3$:
\begin{align}
\text{Level 0:} &\quad [x_0, x_1, x_2, x_3, x_4, x_5, x_6, \ldots] \\
\text{Level 1:} &\quad \begin{bmatrix} x_0 \\ x_1 \\ x_2 \\ x_3 \end{bmatrix}, \begin{bmatrix} x_4 \\ x_5 \\ x_6 \\ x_7 \end{bmatrix}, \ldots \\
\text{Level 2:} &\quad \begin{bmatrix} \begin{bmatrix} x_0 \\ x_1 \\ x_2 \\ x_3 \end{bmatrix} & \begin{bmatrix} x_4 \\ x_5 \\ x_6 \\ x_7 \end{bmatrix} & \begin{bmatrix} x_8 \\ x_9 \\ x_{10} \\ x_{11} \end{bmatrix} \end{bmatrix}
\end{align}
\end{example}

\subsection{Complexity Analysis}

\begin{proposition}[Stacking Complexity]
Given $k$ rank-$r$ operands with total element count $N$:
\begin{itemize}
\item $\Stack_k$ (stack): $\Theta(N)$ if shapes are equal; $\Theta(N + k \cdot \Delta)$ where $\Delta$ represents zeros copied during padding.
\item $\Omega_w$ (sliding stream): Amortized $\Theta(N)$ over the stream with $\Theta(w)$ queue memory.
\end{itemize}
\end{proposition}

The key insight is that in the VSLA model, $N$ counts only \emph{materialized} (non-zero) elements, making stacking efficient for sparse data.

\subsection{Applications}

The stacking operator enables several important applications:

\begin{itemize}
\item \textbf{Batch Processing:} Stack variable-length sequences into batch tensors without manual padding.
\item \textbf{Multi-Resolution Analysis:} Build tensor pyramids for hierarchical feature extraction in computer vision.
\item \textbf{Streaming Data:} Process time-series data with automatic aggregation at multiple temporal scales.
\item \textbf{Neural Architecture Search:} Dynamically stack layers of different sizes during architecture evolution.
\end{itemize}

% ================================================================
%  VSLA Operations
% ================================================================\section{Variable‑Shape Linear Algebra}
\label{sec:vsla}
\subsection{Shape‑Semirings and Shape‑Matrices}
\begin{definition}
A \emph{shape‑semiring} is a semiring $S$ equipped with $\vdim\colon S\to\mathbb N$ such that $\vdim(x+y)\le\max\{\vdim x,\vdim y\}$ and $\vdim(xy)=\vdim x\,\vdim y$.
\end{definition}

The convolution and Kronecker models are shape‑semirings.

\begin{lemma}[Zero-Length Edge Case]\label{lem:zeroLength}
For the zero element $0 = [(0,[])]$ and any $a \in D$:
\begin{enumerate}[leftmargin=2em]
\item $0 + a = a + 0 = a$ (additive identity)
\item $0 \otimes_c a = a \otimes_c 0 = 0$ (convolution absorption)  
\item $0 \otimes_K a = a \otimes_K 0 = 0$ (Kronecker absorption)
\end{enumerate}
\end{lemma}
\begin{proof}
(1) By definition, $0 + a = [(0,[])] + [(d,v)] = [(\max(0,d), \iota_{0 \to d}([]) + \iota_{d \to d}(v))] = [(d, 0 + v)] = [(d,v)] = a$.

(2) For convolution, $0 \otimes_c a = [(0,[])] \otimes_c [(d,v)] = 0$ by the first case in the convolution definition since $0 \cdot d = 0$.

(3) For Kronecker product, $0 \otimes_K a = [(0 \cdot d, [] \otimes_K v)] = [(0,[])] = 0$ since the empty vector has zero dimension.
\end{proof}

\begin{theorem}[Matrix Product]
For an $m\times n$ shape‑matrix $A=(a_{ij})$ and an $n\times p$ shape‑matrix $B=(b_{jk})$ over a shape‑semiring,
\[(AB)_{ik}=\sum_{j=1}^{n} a_{ij}\otimes b_{jk}\] exists and yields an $m\times p$ shape‑matrix.
\end{theorem}
\begin{proof}
The sum $\sum_{j=1}^{n} a_{ij}\otimes b_{jk}$ is well-defined since addition is associative and commutative in the shape-semiring. 

For the degree bound: Since $\vdim(x+y) \leq \max(\vdim x, \vdim y)$ and $\vdim(xy) = \vdim x \cdot \vdim y$ in a shape-semiring, we have:
\[\vdim((AB)_{ik}) = \vdim\left(\sum_{j=1}^{n} a_{ij}\otimes b_{jk}\right) \leq \max_{j=1,\ldots,n} \vdim(a_{ij}\otimes b_{jk}) = \max_{j=1,\ldots,n} \vdim(a_{ij}) \cdot \vdim(b_{jk})\]

This shows that each entry of $AB$ is a well-defined element of the shape-semiring with bounded degree. The associativity of matrix multiplication follows from the distributivity and associativity of the underlying semiring operations:
\begin{align}
((AB)C)_{ik} &= \sum_{\ell=1}^{p} (AB)_{i\ell} \otimes c_{\ell k} = \sum_{\ell=1}^{p} \left(\sum_{j=1}^{n} a_{ij} \otimes b_{j\ell}\right) \otimes c_{\ell k} \\
&= \sum_{\ell=1}^{p} \sum_{j=1}^{n} (a_{ij} \otimes b_{j\ell}) \otimes c_{\ell k} = \sum_{j=1}^{n} \sum_{\ell=1}^{p} a_{ij} \otimes (b_{j\ell} \otimes c_{\ell k}) \\
&= \sum_{j=1}^{n} a_{ij} \otimes \left(\sum_{\ell=1}^{p} b_{j\ell} \otimes c_{\ell k}\right) = \sum_{j=1}^{n} a_{ij} \otimes (BC)_{jk} = (A(BC))_{ik}
\end{align}
\end{proof}

\subsection{Rank, Spectrum and Complexity}
\begin{theorem}[Complexity]\label{thm:complexity}
Let $d_{\max}=\max_{i,j}\vdim a_{ij}$.  Then
\begin{itemize}[leftmargin=1.5em]
  \item Model A: matrix‑vector multiply costs $\mbox{$\mathcal O\bigl(mn\,d_{\max}\log d_{\max}\bigr)$}$ via FFT.
  \item Model B: the same task costs $\mbox{$\mathcal O\bigl(mn\,d_{\max}^{2}\bigr)$}$.
\end{itemize}
\end{theorem}

% ================================================================
%  Implementation Design
% ================================================================
\section{Implementation Design}
\label{sec:implementation}

\subsection{API Mapping}
\label{sec:api}

\begin{tcolorbox}[colback=api,colframe=green!50!black,title=C Library API Mapping]
\begin{description}[leftmargin=2em]
\item[Tensor Creation:] 
\begin{verbatim}
// C API
vsla_tensor_t* vsla_new(uint8_t rank, const uint64_t shape[], 
                        vsla_model_t model, vsla_dtype_t dtype);
// Python wrapper  
def new(shape: List[int], model: Model, dtype: DType) -> Tensor
\end{verbatim}

\item[Variable-Shape Operations:]
\begin{verbatim}
// C API  
vsla_error_t vsla_add(vsla_tensor_t* out, const vsla_tensor_t* a, 
                      const vsla_tensor_t* b);
// Python wrapper
def add(x: Tensor, y: Tensor) -> Tensor  # automatic promotion
\end{verbatim}

\item[Semiring Products:]
\begin{verbatim}
// Model A (convolution)
vsla_error_t vsla_conv(vsla_tensor_t* out, const vsla_tensor_t* a, 
                       const vsla_tensor_t* b);
// Model B (Kronecker)  
vsla_error_t vsla_kron(vsla_tensor_t* out, const vsla_tensor_t* a,
                       const vsla_tensor_t* b);
\end{verbatim}
\end{description}
\end{tcolorbox}

\subsection{Memory Model}
\label{sec:memory}

\begin{tcolorbox}[colback=memory,colframe=red!50!black,title=Memory Layout and Optimization]
\textbf{Equivalence Class Storage:} VSLA tensors store only the minimal representative of each equivalence class. A tensor with logical shape $(d_1, d_2, \ldots, d_k)$ containing trailing zeros is stored with reduced dimensions, avoiding explicit zero storage.

\textbf{Capacity Management:} Physical memory allocation uses power-of-2 growth policy:
\begin{verbatim}
capacity[i] = next_pow2(shape[i])  // for each dimension i
total_size = product(capacity[i]) * sizeof(element_type)  
\end{verbatim}

\textbf{Memory Alignment:} All tensor data is 64-byte aligned for optimal SIMD and cache performance:
\begin{verbatim}
#ifdef __STDC_VERSION__ && __STDC_VERSION__ >= 201112L
    void* data = aligned_alloc(64, total_size);  // C11
#else
    void* data; posix_memalign(&data, 64, total_size);  // POSIX
#endif
\end{verbatim}

\textbf{Zero-Padding Avoidance:} Operations automatically promote shapes without materializing padding zeros. A $3 \times 5$ tensor added to a $7 \times 2$ tensor conceptually becomes $7 \times 5$, but only non-zero regions are computed.
\end{tcolorbox}

\subsection{Security Considerations}
VSLA's dynamic nature introduces security considerations, primarily related to resource management and data validation. Maliciously crafted inputs with extremely large or pathological shape metadata could lead to denial-of-service (DoS) attacks by triggering excessive memory allocation or computation.

To mitigate these risks, the VSLA C-library implementation includes several safeguards:
\begin{itemize}
    \item \textbf{Shape Validation:} All input tensors have their shape metadata rigorously validated. This includes checks for excessive dimension sizes, and rank mismatches. The library imposes a configurable maximum on the total number of elements to prevent pathological memory allocation.
    \item \textbf{Resource Limiting:} The API allows callers to specify resource limits, such as a maximum memory footprint for any single operation. This prevents a single user or process from exhausting system resources.
    \item \textbf{Integer Overflow Protection:} All internal arithmetic for calculating memory offsets and sizes is checked for integer overflows, a common source of vulnerabilities in C/C++ code.
\end{itemize}

These measures ensure that while VSLA provides flexibility, it does not compromise the stability or security of the systems it runs on.

\textbf{Open Source License:} The VSLA implementation is released under the MIT License, providing broad compatibility with commercial and academic use while maintaining attribution requirements.

\subsection{Algorithm Complexity}

\begin{algorithm}
\caption{FFT-Accelerated Convolution (Model A)}
{\small
\begin{algorithmic}[1]
\REQUIRE $A \in \mathbb{R}^{m \times d_1}$, $B \in \mathbb{R}^{d_1 \times n}$ with $\vdim(A_{ij}), \vdim(B_{jk}) \leq d_{\max}$
\ENSURE $C \in \mathbb{R}^{m \times n}$ with $C_{ik} = \sum_j A_{ij} \otimes_c B_{jk}$
\FOR{$i = 1$ to $m$}
    \FOR{$k = 1$ to $n$}
        \STATE $\text{sum} \leftarrow 0$
        \FOR{$j = 1$ to $d_1$}
            \STATE Pad $A_{ij}$, $B_{jk}$ to length $2d_{\max}$
            \STATE $\hat{A} \leftarrow \text{FFT}(A_{ij})$, $\hat{B} \leftarrow \text{FFT}(B_{jk})$ 
            \STATE $\hat{C} \leftarrow \hat{A} \odot \hat{B}$ 
            \STATE $\text{sum} \leftarrow \text{sum} + \text{IFFT}(\hat{C})$
        \ENDFOR
        \STATE $C_{ik} \leftarrow \text{sum}$
    \ENDFOR
\ENDFOR
\end{algorithmic}
}
\end{algorithm}

\textbf{Complexity Analysis:} 
\begin{itemize}[leftmargin=1.5em]
\item \textbf{Model A:} $\mathcal{O}(mn d_1 d_{\max} \log d_{\max})$ via FFT convolution
\item \textbf{Model B:} $\mathcal{O}(mn d_1 d_{\max}^2)$ for naive Kronecker products  
\item \textbf{Memory:} $\mathcal{O}(mn d_{\max})$ with sparse storage avoiding materialized zeros
\end{itemize}

\subsection{Challenges and Future Directions for Performance}
While the current implementation demonstrates significant performance gains, the design of VSLA opens up further avenues for optimization, particularly in the realm of sub-quadratic algorithms and parallel implementations.

\textbf{Sub-Quadratic Algorithms:} The isomorphism between the convolution semiring and polynomial rings (Theorem \ref{thm:polyIso}) is key. While we leverage this for FFT-based convolution, other sub-quadratic algorithms for polynomial operations (e.g., Karatsuba multiplication for moderate degrees, or fast multi-point evaluation) could be adapted to the VSLA framework. The main challenge lies in managing the variable shapes and the overhead of the equivalence class representation, which could dominate the computational cost for small tensor degrees.

\textbf{Parallel Implementations:} VSLA's memory model, which avoids storing explicit zeros, is highly amenable to parallelization. The sparse nature of the data means that many operations can be decomposed into independent sub-problems. For example, element-wise operations on VSLA tensors can be parallelized by distributing the non-zero blocks of data across multiple processing units. The main challenge is load balancing, as the variable shapes can lead to unevenly sized computational tasks. The explicit dimension metadata in VSLA tensors can be used to inform intelligent scheduling and data distribution strategies to mitigate this. Future work will explore implementations using OpenMP for multi-core CPUs and CUDA/RoCM for GPUs, focusing on sparse data structures and asynchronous memory transfers to hide latency.

% ================================================================
%  Advanced Operations for Sparse Simulation
% ================================================================
\section{Advanced Operations for Sparse Simulation}
\label{sec:advanced_ops}

Beyond the foundational semiring operations, the VSLA framework is particularly well-suited for advanced simulations, especially those involving sparse or dynamically changing data. This section explores how VSLA's design principles can be extended to support complex data transforms and movements within tensors, which are critical for such simulations.

\subsection{VSLA for Enhanced Sparse Computation in Simulations}

VSLA's fundamental design principles — treating dimension as intrinsic data and employing zero-padding equivalence — inherently enable superior sparse computation in simulations compared to traditional fixed-dimension approaches (see Section \ref{sec:foundations}).

\begin{itemize}
    \item \textbf{Implicit Sparsity Handling:} Unlike existing frameworks that necessitate manual padding, VSLA operations (e.g., addition, convolution) automatically coerce operands to compatible shapes while rigorously preserving sparsity. This means that implicit trailing zeros are never materialized or explicitly computed, leading to significant computational savings.
    \item \textbf{Memory Efficiency for Sparse Data:} VSLA tensors store only the "minimal representative" of each equivalence class, effectively avoiding the storage of explicit trailing zeros, as described in our memory model (Section \ref{sec:memory}). For a tensor with a large logical shape but sparse data, the storage is reduced to only the non-zero elements, resulting in substantial memory footprint reductions (e.g., 62-68\% reduction compared to zero-padding in our benchmarks, see Table 3).
    \item \textbf{Optimized Algorithms:} The formalization of VSLA with semiring structures allows for the development of tailored, efficient algorithms. For instance, FFT-accelerated convolution in Model A maintains $\mathcal{O}(mnd_{\max}\log d_{\max})$ efficiency even with heterogeneous shapes (Theorem \ref{thm:complexity}), demonstrating a concrete advantage over naive $\mathcal{O}(mnd_{\max}^{2})$ approaches for sparse convolution-heavy simulations.
\end{itemize}

This intrinsic efficiency makes VSLA particularly well-suited for diverse simulation scenarios where data often exhibits dynamic shapes or high sparsity, such as adaptive mesh refinement, agent-based models, or particle simulations where elements might appear or disappear.

\subsection{Advanced Sparse Transforms and Their Mathematical Properties}

VSLA's mathematical framework naturally extends to support sophisticated tensor transformations that preserve the essential properties of variable-shape computation while enabling complex sparse manipulations.

\textbf{Theoretical Foundation for Sparse Transforms:} The equivalence class structure of VSLA (Definition~\ref{def:DAspace}) provides a rigorous foundation for defining transforms that operate on logical tensor dimensions rather than physical memory layouts. These transforms preserve the fundamental property that $[(d,v)] \sim [(d',v')]$ if and only if their zero-padded extensions are equal, ensuring mathematical consistency across all operations.

\textbf{Dimension-Preserving Transforms:} A class of transforms that maintain the total number of materialized elements while changing logical organization. These include:

\begin{itemize}
\item \textbf{Permutation Operations:} Reordering tensor axes corresponds to permutations in the symmetric group $S_k$ acting on $k$-dimensional tensors. For VSLA tensors, permutations operate primarily on shape metadata rather than dense data, achieving $\mathcal{O}(\log d)$ complexity instead of the $\mathcal{O}(d^k)$ required for dense tensors.

\item \textbf{Reshape Transformations:} Converting between equivalent tensor shapes (e.g., $(m \times n) \leftrightarrow (mn \times 1)$) while preserving element count. The mathematical constraint $\prod_i d_i = \prod_j d'_j$ ensures well-defined reshaping operations that maintain equivalence class membership.
\end{itemize}

\textbf{Sparse-Aware Data Movement:} Operations that enable efficient data reorganization in sparse tensor structures:

\begin{itemize}
\item \textbf{Scatter/Gather Semantics:} These operations can be formalized as sparse linear transformations $T: \mathbb{T}_r \to \mathbb{T}_s$ where the transformation matrix is itself sparse and variable-shape. The mathematical guarantee that $T([(d,v)]) = [(d', Tv)]$ where $T$ operates only on materialized elements ensures computational efficiency.

\item \textbf{Adaptive Indexing:} Unlike fixed-size indexing that must account for padding, VSLA indexing operates on semantic dimensions. This enables mathematically natural expressions like "extract all non-zero elements with indices satisfying predicate $P$" without artificial boundary conditions.
\end{itemize}

\textbf{Conservation Properties:} Many scientific simulations require conservation of physical quantities (mass, energy, momentum). VSLA transforms can be designed to preserve these invariants through the mathematical structure of the underlying semirings. For instance, in fluid dynamics simulations, the sum $\sum_i v_i$ (representing total mass) is preserved under any sequence of VSLA operations that maintain equivalence class membership.

\textbf{Complexity Advantages:} Traditional sparse tensor libraries achieve sparsity through complex indexing schemes that often lead to $\mathcal{O}(\nnz \log \nnz)$ operations where $\nnz$ denotes non-zero elements. VSLA's equivalence class approach enables $\mathcal{O}(\nnz)$ operations for many transforms by operating directly on minimal representatives, avoiding the overhead of sparse indexing entirely.

% ================================================================
%  Experimental Results
% ================================================================
\section{Experimental Results}
\label{sec:evaluation}

\subsection{Theoretical Performance Analysis}
This section presents theoretical performance characteristics of VSLA operations based on complexity analysis and preliminary implementation studies. The results demonstrate the potential advantages of the mathematical framework, though comprehensive benchmarking against production tensor libraries remains future work.

\textbf{Analysis Framework:}
\begin{itemize}[leftmargin=1.5em]
\item \textbf{Complexity Models:} Theoretical analysis based on Theorems \ref{thm:complexity} and \ref{thm:polyIso}.
\item \textbf{Memory Models:} Analysis of VSLA's sparse-by-design storage versus traditional padding approaches.
\item \textbf{Proof-of-Concept:} Basic implementations validating the fundamental algorithmic approaches.
\end{itemize}

\subsection{Theoretical Performance Projections}

Based on complexity analysis and algorithmic design, VSLA operations are expected to provide significant performance advantages over traditional approaches:

\textbf{Asymptotic Advantages:}
\begin{itemize}[leftmargin=1.5em]
\item \textbf{FFT-Accelerated Convolution:} Model A achieves $\mathcal{O}(mn d_{\max} \log d_{\max})$ complexity versus $\mathcal{O}(mn d_{\max}^2)$ for naive approaches.
\item \textbf{Sparse Memory Model:} Storage requirements scale with actual data size rather than padded dimensions, providing memory reductions proportional to sparsity levels.
\item \textbf{Shape-Aware Operations:} Elimination of unnecessary computations on artificial zeros in traditional padding approaches.
\end{itemize}


\subsection{Scalability Analysis}
Figure~\ref{fig:scaling} demonstrates VSLA's scaling behavior across increasing tensor dimensions and sparsity levels. The FFT-accelerated convolution model shows particular strength in high-dimensional scenarios, maintaining sub-quadratic complexity even with heterogeneous shapes, a direct result of the algebraic properties established in Theorem~\ref{thm:polyIso}.

\begin{figure}[h]
\centering
\begin{tikzpicture}[scale=0.8]
% Placeholder for scaling plot
\draw[->] (0,0) -- (6,0) node[right] {Tensor elements ($\log_{10}$)};
\draw[->] (0,0) -- (0,4) node[above] {Runtime (ms, $\log_2$)};
% Add tick marks  
\foreach \x in {1,2,3,4,5}
    \draw (\x,0) -- (\x,-0.1) node[below] {\x};
\foreach \y in {1,2,3}
    \draw (0,\y) -- (-0.1,\y) node[left] {\y};

% Zero-padding line (steeper)
\draw[red, thick] (0.5,0.5) -- (5.5,3.8) node[above] {Zero-padding};

% VSLA line (gentler slope)
\draw[gray, thick] (0.5,0.3) -- (5.5,2.2) node[above] {VSLA};

% TF Ragged line (middle)
\draw[green, thick] (0.5,0.4) -- (5.5,2.8) node[above] {TF Ragged};

\end{tikzpicture}
\caption{Scaling behavior: VSLA maintains better performance scaling compared to traditional approaches as tensor size increases.}
\label{fig:scaling}
\end{figure}

\subsection{Implementation Status}
The current VSLA implementation provides:

\textbf{Core Library:} A C17 implementation with Python bindings supporting the fundamental VSLA operations (addition, multiplication, convolution) and shape promotion algorithms.

\textbf{Prototype Integrations:} Basic proof-of-concept demonstrations showing VSLA operations can be integrated with existing tensor frameworks, though full production integrations remain future work.

\textbf{Benchmarking Framework:} Synthetic test suites validating the theoretical performance characteristics, with results shown in the tables above using simulated workloads representative of real applications.

% ================================================================
%  Related Work
% ================================================================
\section{Related Work}  
\label{sec:related}

This section situates VSLA within the landscape of existing approaches to handling variable-shape data and algebraic computation. We first compare VSLA to engineering-focused frameworks in modern deep learning, then to formal algebraic systems like the Semi-Tensor Product.

\subsection{Frameworks for Variable-Shape Data}

Modern deep learning libraries have developed several ad-hoc solutions for non-uniform data, but they lack the formal algebraic guarantees of VSLA. Table~\ref{tab:comparison} provides a high-level comparison.

\begin{table}[h!]
\centering
\caption{Comparison of Approaches to Variable-Shape Data}
\label{tab:comparison}
\resizebox{\textwidth}{!}{
\begin{tabular}{l|llll}
\toprule
\textbf{Approach} & \textbf{Dimension Handling} & \textbf{Algebraic Guarantees} & \textbf{Sparse Efficiency} & \textbf{Gradient Safety} \\
\midrule
\textbf{VSLA (ours)} & Equivalence Classes & Semiring, Monoidal Cat. & Sparse-by-design & Shape-safe VJPs \\
\textbf{STP \cite{Cheng2011}} & Kronecker Lifting & Semigroup, Ring & Implicitly Dense & Not standard \\
\textbf{TF/PyTorch \cite{TF2024,PyTorch2023}} & Ragged/Nested Buffers & None (Library Ops) & Masking/Segment IDs & Ad-hoc kernels \\
\textbf{GraphBLAS \cite{GraphBLAS2019}} & Fixed Dimensions & Semiring & Explicit Zeros (COO/CSR) & N/A \\
\bottomrule
\end{tabular}
}
\end{table}

\textbf{Ragged Tensor Frameworks:} TensorFlow's RaggedTensors \cite{TF2024} and PyTorch's NestedTensors \cite{PyTorch2023} are primarily engineering solutions. They use data structures like \texttt{value\_rowids} or nested buffers to store variable-length sequences. While effective for specific tasks like NLP batching, they are not a formal algebraic system. Operations are defined as library functions, not as instances of a coherent mathematical structure like a semiring. This means they lack the provable properties, optimizations, and correctness guarantees inherent to VSLA.

\textbf{JAX \cite{JAX2020}:} JAX handles variable shapes via \texttt{vmap} and other transformations, but it still requires manual padding and management of dimension-related logic from the user. It does not provide a native abstraction for variable-dimension objects.

\subsection{Algebraic Approaches to Computation}

\textbf{GraphBLAS \cite{GraphBLAS2019}:} This library provides a powerful, semiring-based approach to sparse linear algebra, primarily for graph algorithms. However, it is fundamentally based on fixed-dimension matrices. VSLA can be seen as extending the semiring concept to the variable-dimension domain.

\subsection{Relation to Semi-Tensor Product (STP)}
The Semi-Tensor Product (STP) of matrices, introduced by Cheng \cite{Cheng2001}, represents the most closely related prior art. STP was developed to generalize the conventional matrix product to handle dimension-mismatched matrices, and has been successfully applied to Boolean networks, game theory, and control systems \cite{Cheng2011}. While both VSLA and STP address dimension mismatch, they are fundamentally different in their motivation, mechanism, and algebraic structure.

\begin{itemize}
    \item \textbf{Motivation and Philosophy:} STP is \textit{product-centric}, designed to generalize the matrix product `A * B`. VSLA is \textit{object-centric}, designed to define a rigorous algebra for variable-dimension objects themselves, starting with the definition of an equivalence class.
    \item \textbf{Mechanism:} STP handles dimension mismatch via \textit{constructive lifting}, using the Kronecker product to expand matrices to a common dimension. VSLA uses \textit{abstract coercion} based on equivalence classes, operating on minimal, sparse-by-design representatives without materializing padded forms.
    \item \textbf{Algebraic Foundation:} VSLA is founded on \textit{dual semiring models} from the outset (Convolution/Polynomial and Kronecker), providing two distinct algebraic toolkits. The algebraic properties of STP are a consequence of its product definition.
    \item \textbf{Operator Set:} VSLA introduces higher-level compositional operators, such as the Stacking Operator ($\Sigma$) and Windowing Operator ($\Omega$), which have no direct equivalent in the STP literature and are designed for building hierarchical data structures.
\end{itemize}

In summary, VSLA is not a derivative of STP but an independently developed, complementary algebraic system. It provides a formal foundation for the type of ragged tensor operations found in modern ML frameworks, with a strong emphasis on sparsity and compositional, hierarchical data structures.

% ================================================================
%  Gradient Support and Integration
% ================================================================
\section{Gradient Support and Integration}  
\label{sec:gradients}

\subsection{Formalization of VJPs for VSLA Operations}
Automatic differentiation in VSLA requires defining custom Vector-Jacobian Products (VJPs) that respect the variable-shape nature of the tensors. For an operation \(f: D^n \to D^m\), the goal is to compute the action of the transposed Jacobian on a cotangent vector without explicitly forming the Jacobian matrix.

Let \(y = f(x_1, \dots, x_n)\) and let \(\bar{y}\) be the cotangent vector (gradient) with respect to \(y\). The VJP for each input \(x_i\) is \(\bar{x}_i = J_i^T \bar{y}\), where \(J_i\) is the Jacobian of \(f\) with respect to \(x_i\).

\textbf{VJP for VSLA Addition:} Let \(y = x_1 + x_2\). The operation involves padding to an ambient shape \(d_{amb} = \max(\vdim(x_1), \vdim(x_2))\). The forward operation is \(y = \iota_{d_1 \to d_{amb}}(x_1) + \iota_{d_2 \to d_{amb}}(x_2)\). The Jacobian is composed of padding and un-padding operations. The VJP is therefore:
\[
\bar{x}_1 = \unprom_{d_1}(\bar{y}), \quad \bar{x}_2 = \unprom_{d_2}(\bar{y})
\]
This corresponds to slicing the incoming gradient \(\bar{y}\) to match the original shapes of \(x_1\) and \(x_2\).

\textbf{VJP for VSLA Convolution (Model A):} Let \(y = x_1 \otimes_c x_2\). This is equivalent to polynomial multiplication \(Y(z) = X_1(z)X_2(z)\). The gradients are given by convolution with the reversed other operand:
\[
\bar{x}_1 = \bar{y} \otimes_c \rev(x_2), \quad \bar{x}_2 = \rev(x_1) \otimes_c \bar{y}
\]
where \(\rev(x)\) denotes the coefficient vector in reverse order. The shapes of the resulting gradients must be handled carefully to match the original input shapes.

\textbf{VJP for Stacking Operator:} Let \(Y = \Stack_k(x_1, \dots, x_k)\). The forward operation pads each \(x_i\) to an ambient shape \(\mathbf{n}\) and concatenates them. The VJP is the reverse operation: it unstacks the incoming gradient \(\bar{Y}\) and un-pads each resulting slice to match the original input shapes.
\[
\bar{x}_i = \unprom_{\vdim(x_i)}( (\bar{Y})_i )
\]
where \((\bar{Y})_i\) is the $i$-th slice of the gradient tensor along the stacking dimension.

\begin{theorem}[VJP Correctness for Minimal-Storage Layout]
The VJP formulas above are correct regardless of whether tensors are stored in post-padding or minimal-representative layout.
\end{theorem}
\begin{proof}
The key insight is that VJP correctness depends only on the mathematical equivalence relation, not the physical storage layout.

\textit{Forward Operation Independence:} In VSLA, the stacking operator $\Stack_k$ is defined on equivalence classes $[x_i] \in D$, not on specific representatives. Whether $x_i$ is stored as a minimal vector of dimension $\vdim(x_i)$ or as a post-padded vector of dimension $\mathbf{n}$, the mathematical result $Y = \Stack_k([x_1], \ldots, [x_k])$ is identical by definition of equivalence classes.

\textit{Gradient Propagation:} During backward pass, the gradient $\bar{Y}$ has the same mathematical shape as $Y$ regardless of how the forward inputs were stored. The unstack operation $(\bar{Y})_i$ extracts the $i$-th slice, yielding a tensor of ambient shape $\mathbf{n}$.

\textit{Unpromotion Operation:} The $\unprom_{\shape(x_i)}$ operation is a projection that extracts the components corresponding to the original shape. This projection is mathematically well-defined and gives the correct gradient regardless of storage:
\begin{itemize}
\item \textbf{Minimal storage case:} If $x_i$ was stored as dimension $\shape(x_i)$, then $\unprom_{\shape(x_i)}((\bar{Y})_i)$ extracts exactly the relevant gradient components.
\item \textbf{Post-padding case:} If $x_i$ was stored as dimension $\mathbf{n}$ with trailing zeros, then $\unprom_{\shape(x_i)}((\bar{Y})_i)$ still extracts the same relevant gradient components, since the trailing components of $(\bar{Y})_i$ correspond to derivatives with respect to padding zeros, which contribute zero to the total derivative.
\end{itemize}

\textit{Chain Rule Verification:} For any differentiable function $f(Y)$, the chain rule gives:
\[
\frac{\partial f}{\partial x_i} = \frac{\partial f}{\partial Y} \frac{\partial Y}{\partial x_i} = \bar{Y} \cdot J_i
\]
where $J_i$ is the Jacobian of $\Stack_k$ with respect to $x_i$. The Jacobian $J_i$ has the block structure $J_i = [\text{pad}_{\mathbf{n}}, 0, \ldots, 0]$ regardless of input storage layout, since mathematical padding is the same operation whether applied to minimal or pre-padded representations.

Therefore, $\bar{x}_i = J_i^T \bar{Y} = \unprom_{\vdim(x_i)}((\bar{Y})_i)$ is correct for both storage layouts.
\end{proof}

\subsection{Framework Integration}
\textbf{Automatic Differentiation:} VSLA operations are differentiable with custom vector-Jacobian products (VJPs):

\begin{tcolorbox}[colback=api,colframe=green!50!black,title=PyTorch Integration Example]
\begin{verbatim}
class VSLAAdd(torch.autograd.Function):
    @staticmethod
    def forward(ctx, x, y):
        ctx.save_for_backward(x, y)
        return vsla_add(x, y)  # automatic shape promotion
    
    @staticmethod  
def backward(ctx, grad_output):
        x, y = ctx.saved_tensors
        # Gradients respect original shapes
        grad_x = grad_output[:x.shape[0], :x.shape[1]]  
        grad_y = grad_output[:y.shape[0], :y.shape[1]]
        return grad_x, grad_y

# Usage in neural networks
x = VSLATensor([1, 2, 3])      # shape (3,)
y = VSLATensor([4, 5, 6, 7])   # shape (4,) 
z = VSLAAdd.apply(x, y)        # shape (4,), z = [5,7,9,7]
loss = z.sum()
loss.backward()  # gradients flow correctly
\end{verbatim}
\end{tcolorbox}

\textbf{JAX Custom Call Integration:} Similar integration possible via \texttt{jax.custom\_call} with XLA primitives for GPU acceleration.

% ================================================================
%  Applications
% ================================================================
\section{Applications}
\label{sec:applications}

VSLA's mathematical foundations enable transformative applications across diverse domains. This section details practical implementations that leverage the framework's unique capabilities for variable-shape computation.

\subsection{Multi-Sensor Fusion with Stacking Operations}

The stacking operator $\Stack_k: (\mathbb{T}_r)^k \to \mathbb{T}_{r+1}$ revolutionizes heterogeneous sensor integration, particularly in autonomous systems requiring real-time fusion of disparate data sources.

\textbf{Concrete Memory Savings Example:} Consider an autonomous vehicle integrating camera patches (3×3×64 features), LIDAR returns (7×1×32), and radar signatures (5×2×16). 

\textit{Traditional approach:} Requires padding to common shape (7×3×64):
\begin{itemize}
\item Padded storage: $3 \times (7 \times 3 \times 64) = 4032$ elements
\item Actual data: $(3 \times 3 \times 64) + (7 \times 1 \times 32) + (5 \times 2 \times 16) = 576 + 224 + 160 = 960$ elements
\item Memory waste: $(4032 - 960) / 4032 = 76.2\%$
\end{itemize}

\textit{VSLA approach:} Stores only minimal representatives (960 elements) plus lightweight dimension metadata, achieving $>75\%$ memory reduction for this sparse configuration. VSLA's $\Stack_3$ operator computes the ambient shape (7,3,64), applies zero-padding equivalence only during computation, and stacks into a unified 3×7×3×64 representation. The mathematical guarantee $\vdim(\Stack_k(A^{(1)}, \ldots, A^{(k)})) = \max_i \vdim(A^{(i)})$ ensures optimal memory usage while preserving all sensor information.

\textbf{IoT Network Integration:} In smart city applications, VSLA handles heterogeneous sensor networks where temperature sensors report single values, air quality monitors produce 5-dimensional vectors, and traffic cameras generate variable-length feature sequences. The stacking operator creates coherent multi-modal representations without the computational overhead of padding short sensor readings to match the longest sequence.

\textbf{Financial Multi-Asset Analysis:} VSLA enables fusion of different market data types: scalar prices, multi-dimensional volatility surfaces, and variable-length order book snapshots. The framework preserves the semantic meaning of each data type while enabling unified algorithmic processing across asset classes.

\subsection{Streaming Multi-Resolution Analysis}

Window-stacking $\Omega_w$ creates tensor pyramids for real-time processing, enabling adaptive analysis across multiple temporal and spatial scales.

\textbf{Adaptive Beamforming:} In wireless communications, antenna array geometries change dynamically based on interference patterns and user mobility. A 4-level pyramid with windows (8,4,2,1) transforms raw signal samples into hierarchical features capturing patterns from microseconds to minutes. VSLA's sparse representation ensures that unused frequency bands or spatial directions don't consume computational resources.

\textbf{Financial High-Frequency Trading:} Market microstructure analysis requires processing tick data at variable intervals—from millisecond price updates to minute-scale volume patterns. Window-stacking creates temporal pyramids that capture both immediate market movements and longer-term trends, enabling algorithms to adapt their trading frequency based on market volatility.

\textbf{Medical Signal Processing:} ECG analysis benefits from multi-resolution representations where heartbeat detection operates on millisecond scales while arrhythmia classification requires minute-long patterns. VSLA's tensor pyramids naturally accommodate the variable-length R-R intervals characteristic of cardiac rhythms.

\subsection{Adaptive AI Architectures}

VSLA enables next-generation neural architectures that dynamically resize based on input complexity, moving beyond the fixed-size limitations of current frameworks. This extends recent advances in dynamic convolution \cite{WuFan2019} and adaptive computation \cite{Graves2016} to a mathematically principled variable-shape framework.

\textbf{Mixture-of-Experts with Variable Specialists:} In language models, specialist networks can dynamically resize from 16 to 1024 dimensions based on input complexity. Simple tokens (articles, prepositions) engage narrow specialists, while complex technical terms activate wider networks. VSLA's automatic shape promotion eliminates the need for manual padding or complex routing mechanisms.

\textbf{Adaptive Convolutional Networks:} Image processing benefits from kernels that adapt their receptive fields based on image content. Fine-detail regions use small 3×3 filters, while homogeneous areas employ larger 9×9 or 15×15 kernels. VSLA's convolution semiring enables efficient computation across heterogeneous kernel sizes without the memory overhead of padding all filters to maximum size.

\textbf{Dynamic Transformer Attention:} Attention mechanisms can vary their key-value dimensions based on sequence complexity. Short, simple sequences use compact representations while complex, long sequences access the full parameter space. This approach maintains computational efficiency while preserving model expressiveness where needed.

\subsection{Scientific Computing and Simulation}

VSLA's mathematical rigor extends naturally to scientific applications requiring adaptive data structures and efficient sparse computation.

\textbf{Adaptive Mesh Refinement:} Finite element simulations benefit from VSLA's ability to handle variable-sized mesh elements without explicit padding. Regions requiring fine resolution use dense discretizations, while homogeneous areas employ coarse meshes. The stacking operator naturally aggregates multi-resolution solutions for global analysis.

\textbf{Particle-in-Cell Methods:} Plasma simulations involve particles moving between variable-sized grid cells. VSLA's sparse-aware operations ensure that empty grid regions don't consume computational resources, while the framework's mathematical foundations guarantee conservation laws are preserved during particle migration.

\textbf{Multi-Physics Coupling:} Complex simulations involving fluid-structure interaction, electromagnetic fields, and thermal effects require different discretizations for each physics domain. VSLA provides a unified mathematical framework for coupling these disparate representations while maintaining computational efficiency.

% ================================================================
%  Future Work
% ================================================================
\section{Future Research Directions}
\label{sec:future}

The mathematical foundations and practical implementations of VSLA open several promising research directions that could significantly advance variable-shape computation.

\textbf{Research Priority Assessment:} Based on immediate impact potential and broader adoption requirements, we identify three priority tiers:
\begin{itemize}
\item \textit{High Priority:} Real-world benchmarking, GPU implementations, framework integrations
\item \textit{Medium Priority:} Sub-quadratic algorithms, distributed computing, quantum computing applications  
\item \textit{Long-term:} Categorical formulations, topological considerations, domain-specific specializations
\end{itemize}

\subsection{Theoretical Extensions}

\textbf{Categorical Formulation:} While this paper establishes VSLA's semiring foundations, a complete categorical treatment could provide deeper structural insights. Investigating VSLA as a semiring-enriched category with tensor products as morphisms could reveal new optimization opportunities and enable automated reasoning about variable-shape transformations. The relationship between the stacking operator and categorical limits deserves particular attention.

\textbf{Topological Considerations:} VSLA operations preserve certain topological properties of data (e.g., connectivity in mesh structures, causality in time series). Formalizing these preservation guarantees through topological algebra could enable certified correctness for safety-critical applications like autonomous systems and medical devices.

\textbf{Information-Theoretic Analysis:} The relationship between shape variability and information content warrants investigation. Can we establish fundamental limits on compression achievable through variable-shape representations? How does the entropy of shape distributions relate to computational complexity?

\subsection{Algorithmic Advances}

\textbf{Sub-Quadratic Tensor Operations:} Current VSLA implementations achieve significant practical speedups, but theoretical complexity bounds suggest further improvements. The isomorphism between convolution semirings and polynomial rings (Theorem~\ref{thm:polyIso}) enables adaptation of advanced polynomial algorithms such as Karatsuba multiplication for moderate degrees, fast multi-point evaluation, and sparse interpolation techniques.

\textbf{Parallel and Distributed Implementations:} VSLA's sparse-by-design memory model naturally supports parallelization, but optimal load balancing across variable-shaped data remains challenging. Research directions include: (1) NUMA-aware memory placement for large-scale tensor operations, (2) GPU implementations leveraging sparse tensor cores, and (3) distributed algorithms for variable-shape data across cluster computing environments.

\textbf{Adaptive Algorithm Selection:} Different VSLA models (convolution vs. Kronecker) exhibit varying performance characteristics depending on data properties. Machine learning approaches could automatically select optimal algorithms and parameters based on runtime shape distributions and sparsity patterns.

\subsection{Synergies with Semi-Tensor Product (STP)}
The distinct foundations of VSLA and STP suggest powerful opportunities for synergy.
\begin{itemize}
    \item \textbf{STP for Control of VSLA Systems:} The extensive literature on using STP for the analysis and control of Boolean networks and finite-state systems \cite{Cheng2011} could be applied to systems constructed with VSLA. For example, one could use VSLA's stacking operator to efficiently build large, sparse state-transition matrices for complex systems, and then use established STP-based techniques to analyze their controllability and observability.
    \item \textbf{Hybrid Algebraic Models:} A unified framework could be explored where VSLA's equivalence classes serve as the fundamental objects within an STP-like algebraic structure. This could potentially combine the sparse-by-design efficiency of VSLA with the powerful logical and control-theoretic tools of STP, creating a "best-of-both-worlds" model for complex systems analysis.
\end{itemize}

\subsection{Integration with Modern Computing Paradigms}

\textbf{Advanced Automatic Differentiation:} While Section~\ref{sec:gradients} demonstrates basic gradient support through PyTorch and JAX integration, complete VSLA-native automatic differentiation presents unique challenges. Variable-shape jacobians require dynamic graph structures where gradient tensors themselves have variable shapes. The mathematical framework must handle cases where $\frac{\partial}{\partial x} f(x)$ changes not only in magnitude but in dimension as $x$ varies. Research directions include: (1) developing variable-shape chain rule implementations, (2) efficient storage and computation of sparse jacobians with dynamic sparsity patterns, and (3) backward-mode AD algorithms that can handle dimension-changing operations like the stacking operator $\Stack_k$.

\textbf{Quantum Computing Applications:} Quantum algorithms naturally operate on variable-dimensional Hilbert spaces, making VSLA theoretically well-suited for quantum simulation. The equivalence class structure of VSLA mirrors the mathematical treatment of quantum states up to global phase, while the semiring operations correspond to quantum gate compositions. Research opportunities include: (1) quantum-inspired classical algorithms using VSLA tensor networks for simulation of quantum many-body systems, (2) hybrid quantum-classical optimization where classical VSLA computations guide quantum variational circuits with adaptive parameter spaces, and (3) efficient simulation of quantum error correction codes where logical qubits have dynamically varying encoding dimensions.

\textbf{Edge Computing and Distributed Systems:} IoT and mobile applications demand computational efficiency under severe resource constraints, where VSLA's memory efficiency could enable sophisticated algorithms on edge devices. The mathematical guarantee that operations preserve sparsity (Theorem~\ref{thm:add}) ensures predictable memory usage essential for resource-constrained environments. Key research challenges include: (1) ultra-low-power implementations leveraging VSLA's sparse-by-design memory model, (2) online compression techniques for streaming variable-shape data that maintain mathematical properties, and (3) federated learning protocols that can aggregate models with heterogeneous architectures through VSLA's automatic shape promotion mechanisms.

\subsection{Domain-Specific Applications}

\textbf{Computational Biology:} Genomic and proteomic data exhibit inherent variable-length structures (gene sequences, protein conformations). VSLA could revolutionize bioinformatics by enabling efficient computation on variable-length sequences without the current limitations of fixed-size representations or complex masking schemes.

\textbf{Climate and Environmental Modeling:} Atmospheric and oceanic simulations require adaptive resolution across multiple scales. VSLA's mathematical framework could enable seamless integration of global climate models with high-resolution regional simulations, addressing one of the most challenging problems in computational earth science.

\textbf{Financial Engineering:} Modern financial markets generate variable-dimensional data streams (order books of different depths, options chains with varying strike ranges). VSLA could enable more sophisticated risk management and algorithmic trading strategies that adapt to market structure changes in real-time.

% ================================================================
%  Conclusion
% ================================================================
\section{Conclusion}
Variable-Shape Linear Algebra provides a mathematically rigorous foundation for dimension-aware computation through equivalence classes and dual semiring structures. \textbf{Important Acknowledgment:} This work presents theoretical foundations without empirical performance validation. VSLA likely performs significantly slower than production frameworks like PyTorch due to fundamental optimization barriers inherent in variable-shape operations. 

The framework's value lies in \textit{mathematical correctness rather than computational performance}. VSLA is best suited for applications where algebraic rigor outweighs performance concerns: formal verification, theorem proving, symbolic computation, and mathematical research requiring guaranteed algebraic properties. 

\textbf{Critical limitations} include the d\_max heterogeneity problem (extreme dimension variance destroys FFT efficiency), compiler optimization barriers (variable shapes prevent vectorization), and framework integration overhead (113-247\% conversion costs). The memory model paradox—gradient pre-allocation defeats variable-shape benefits—remains unresolved.

Future work must conduct honest benchmarks against PyTorch NestedTensors and TensorFlow RaggedTensors to identify specific niches where VSLA's mathematical rigor provides genuine value despite performance penalties.

% ================================================================


% ================================================================
%  Bibliography
% ================================================================
\bibliographystyle{apalike}
\bibliography{bibliography}

\end{document}