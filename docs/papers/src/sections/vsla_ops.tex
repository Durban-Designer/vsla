% ================================================================
%  VSLA Operations
% ================================================================\section{Variable‑Shape Linear Algebra}
\label{sec:vsla}
\subsection{Shape‑Semirings and Shape‑Matrices}
\begin{definition}
A \emph{shape‑semiring} is a semiring $S$ equipped with $\vdim\colon S\to\mathbb N$ such that $\vdim(x+y)\le\max\{\vdim x,\vdim y\}$ and $\vdim(xy)=\vdim x\,\vdim y$.
\end{definition}

The convolution and Kronecker models are shape‑semirings.

\begin{lemma}[Zero-Length Edge Case]\label{lem:zeroLength}
For the zero element $0 = [(0,[])]$ and any $a \in D$:
\begin{enumerate}[leftmargin=2em]
\item $0 + a = a + 0 = a$ (additive identity)
\item $0 \otimes_c a = a \otimes_c 0 = 0$ (convolution absorption)  
\item $0 \otimes_K a = a \otimes_K 0 = 0$ (Kronecker absorption)
\end{enumerate}
\end{lemma}
\begin{proof}
(1) By definition, $0 + a = [(0,[])] + [(d,v)] = [(\max(0,d), \iota_{0 \to d}([]) + \iota_{d \to d}(v))] = [(d, 0 + v)] = [(d,v)] = a$.

(2) For convolution, $0 \otimes_c a = [(0,[])] \otimes_c [(d,v)] = 0$ by the first case in the convolution definition since $0 \cdot d = 0$.

(3) For Kronecker product, $0 \otimes_K a = [(0 \cdot d, [] \otimes_K v)] = [(0,[])] = 0$ since the empty vector has zero dimension.
\end{proof}

\begin{theorem}[Matrix Product]
For an $m\times n$ shape‑matrix $A=(a_{ij})$ and an $n\times p$ shape‑matrix $B=(b_{jk})$ over a shape‑semiring,
\[(AB)_{ik}=\sum_{j=1}^{n} a_{ij}\otimes b_{jk}\] exists and yields an $m\times p$ shape‑matrix.
\end{theorem}
\begin{proof}
The sum $\sum_{j=1}^{n} a_{ij}\otimes b_{jk}$ is well-defined since addition is associative and commutative in the shape-semiring. 

For the degree bound: Since $\vdim(x+y) \leq \max(\vdim x, \vdim y)$ and $\vdim(xy) = \vdim x \cdot \vdim y$ in a shape-semiring, we have:
\[\vdim((AB)_{ik}) = \vdim\left(\sum_{j=1}^{n} a_{ij}\otimes b_{jk}\right) \leq \max_{j=1,\ldots,n} \vdim(a_{ij}\otimes b_{jk}) = \max_{j=1,\ldots,n} \vdim(a_{ij}) \cdot \vdim(b_{jk})\]

This shows that each entry of $AB$ is a well-defined element of the shape-semiring with bounded degree. The associativity of matrix multiplication follows from the distributivity and associativity of the underlying semiring operations:
\begin{align}
((AB)C)_{ik} &= \sum_{\ell=1}^{p} (AB)_{i\ell} \otimes c_{\ell k} = \sum_{\ell=1}^{p} \left(\sum_{j=1}^{n} a_{ij} \otimes b_{j\ell}\right) \otimes c_{\ell k} \\
&= \sum_{\ell=1}^{p} \sum_{j=1}^{n} (a_{ij} \otimes b_{j\ell}) \otimes c_{\ell k} = \sum_{j=1}^{n} \sum_{\ell=1}^{p} a_{ij} \otimes (b_{j\ell} \otimes c_{\ell k}) \\
&= \sum_{j=1}^{n} a_{ij} \otimes \left(\sum_{\ell=1}^{p} b_{j\ell} \otimes c_{\ell k}\right) = \sum_{j=1}^{n} a_{ij} \otimes (BC)_{jk} = (A(BC))_{ik}
\end{align}
\end{proof}

\subsection{Rank, Spectrum and Complexity}
\begin{theorem}[Complexity]\label{thm:complexity}
Let $d_{\max}=\max_{i,j}\vdim a_{ij}$.  Then
\begin{itemize}[leftmargin=1.5em]
  \item Model A: matrix‑vector multiply costs $\mbox{$\mathcal O\bigl(mn\,d_{\max}\log d_{\max}\bigr)$}$ via FFT.
  \item Model B: the same task costs $\mbox{$\mathcal O\bigl(mn\,d_{\max}^{2}\bigr)$}$.
\end{itemize}
\end{theorem}
