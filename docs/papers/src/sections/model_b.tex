% ================================================================
%  Model B: The Kronecker Semiring
% ================================================================
\section{Model
B: The Kronecker Semiring}
\label{sec:modelB}
\subsection{Kronecker Product}
\begin{definition}
For \(v\in\mathbb R^{d_1}\), \(w\in\mathbb R^{d_2}\), let
\[v\otimes_K w := (v_1w_1,\dots,v_1w_{d_2},\,v_2w_1,\dots,v_{d_1}w_{d_2}).\]
Define
\[\bigl[(d_1,v)\bigr]\otimes_K \bigl[(d_2,w)\bigr] := \bigl[(d_1d_2,\,v\otimes_K w)\bigr].\]
\end{definition}

\begin{theorem}
\(\bigl(D,+,\otimes_K,0,1\bigr)\) is a non\-commutative semiring.
\end{theorem}
\begin{proof}
We verify the semiring axioms systematically.

\textit{Additive structure:} $(D,+,0)$ is already a commutative monoid by Theorem~\ref{thm:add}.

\textit{Associativity of $\otimes_K$:} For $a = [(d_1,u)]$, $b = [(d_2,v)]$, $c = [(d_3,w)] $:
\begin{align}
(a \otimes_K b) \otimes_K c &= [(d_1 d_2, u \otimes_K v)] \otimes_K [(d_3,w)] \\
&= [(d_1 d_2 d_3, (u \otimes_K v) \otimes_K w)]
\end{align}
and
\begin{align}
a \otimes_K (b \otimes_K c) &= [(d_1,u)] \otimes_K [(d_2 d_3, v \otimes_K w)] \\
&= [(d_1 d_2 d_3, u \otimes_K (v \otimes_K w)]
\end{align}
Both expressions yield vectors in $\mathbb{R}^{d_1 d_2 d_3}$ with components $(u \otimes_K v \otimes_K w)_{i,j,k} = u_i v_j w_k$ in the lexicographic order, so they are equal.

\textit{Multiplicative identity:} For $1 = [(1,[1])]$ and any $a = [(d,v)] $:
\[1 \otimes_K a = [(1 \cdot d, [1] \otimes_K v)] = [(d, (1 \cdot v_1, 1 \cdot v_2, \ldots, 1 \cdot v_d))] = [(d,v)] = a\]
Similarly, $a \otimes_K 1 = a$.

\textit{Distributivity:} For $a = [(d_1,u)]$, $b = [(d_2,v)]$, $c = [(d_2,w)] $:
\begin{align}
a \otimes_K (b + c) &= [(d_1,u)] \otimes_K [(d_2, v + w)] \\
&= [(d_1 d_2, u \otimes_K (v + w))] \\
&= [(d_1 d_2, (u_1(v_1 + w_1), \ldots, u_1(v_{d_2} + w_{d_2}), \\
&\qquad\qquad u_2(v_1 + w_1), \ldots, u_{d_1}(v_{d_2} + w_{d_2})))] \\
&= [(d_1 d_2, (u \otimes_K v) + (u \otimes_K w))] \\
&= a \otimes_K b + a \otimes_K c
\end{align}
Right distributivity follows similarly.

\textit{Absorption by zero:} $0 \otimes_K a = [(0 \cdot d, \emptyset)] = 0$ and $a \otimes_K 0 = 0$ by the definition of Kronecker product with the empty vector.

\textit{Non-commutativity:} Consider $a = [(2, (1,0))]$ and $b = [(2, (0,1))]$. Then:
\[a \otimes_K b = [(4, (1 \cdot 0, 1 \cdot 1, 0 \cdot 0, 0 \cdot 1))] = [(4, (0,1,0,0))]\]
\[b \otimes_K a = [(4, (0 \cdot 1, 0 \cdot 0, 1 \cdot 1, 1 \cdot 0))] = [(4, (0,0,1,0))]\]
Since $(0,1,0,0) \neq (0,0,1,0)$, we have $a \otimes_K b \neq b \otimes_K a$.
\end{proof}

\paragraph{Practical Implications of Non-Commutativity}
The non-commutativity of the Kronecker product is not merely a theoretical curiosity; it is a crucial feature for many applications. In quantum computing, the state of a multi-qubit system is described by the Kronecker product of single-qubit states, and the order of the product is significant. Similarly, in tensor network methods for simulating many-body quantum systems, such as Matrix Product States (MPS) or Projected Entangled Pair States (PEPS), the non-commutative nature of the Kronecker product is fundamental to capturing the entanglement structure of the system. In these domains, VSLA's Model B provides a natural and efficient framework for representing and manipulating these non-commutative structures, especially when the constituent systems have different dimensions.

\begin{proposition}\label{prop:commCase}
$x\otimes_K y = y\otimes_K x$ iff $\vdim x =1$ or $\vdim y =1$ (i.e. one operand is scalar).
\end{proposition}

\begin{lemma}[Scalar\-Commutation]\label{lem:scalarComm}
If $x=\alpha\,1$ with $\alpha\in\mathbb R$ then $x\otimes_K y = y\otimes_K x$ for all $y\in D$.
\end{lemma}
\begin{proof}
Both products equal $\alpha\,y$ by definition.\end{proof}
